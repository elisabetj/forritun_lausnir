\problemname{Inflation}

This project is an exercise in file processing, lists and tuples.
Write a program that reads information about consumer price index from an input file and prints out various information about the data.
The data in the input files are obtained from the website of Statistics Iceland, \url{https://statice.is/}.
A typical file (\texttt{data9091.txt}), which contains data for the years 1990 and 1991, looks like this:

\begin{verbatim}
    1990M01	139.3
    1990M02	141.5
    1990M03	142.7
    1990M04	143.1
    1990M05	144.4
    1990M06	145.4
    1990M07	146.4
    1990M08	146.8
    1990M09	146.8
    1990M10	147.2
    1990M11	148.2
    1990M12	148.6
    1991M01	149.5
    1991M02	150.0
    1991M03	150.3
    1991M04	151.0
    1991M05	152.8
    1991M06	154.9
    1991M07	156.0
    1991M08	157.2
    1991M09	158.1
    1991M10	159.3
    1991M11	160.0
    1991M12	159.8    
\end{verbatim}
The first column contains year/month in ascending order. The second column contains the index for the corresponding year/month.
The data in an input file SHOULD be read into a list of tuples. 

\section*{Input}
The input is a name of a file to be analyzed, e.g. \texttt{data9091.txt} or any other file name ending with \texttt{.txt}.
Each line in an input file contains a string denoting a year/month and a float denoting the index for the corresponding year/month. 
These two fields are separated by a whitespace.

\section*{Output}
When the input file cannot be opened, no output is generated.  Otherwise, the output consists of the following information:
\begin{enumerate}
    \item One tuple \texttt{(y, i)} in line for each line in the input file, where \texttt{y} is year/month (a string) and \texttt{i} is the corresponding index (a float).
    \item One tuple \texttt{(y, f, l)} in line for each year in the input file, where \texttt{y} is a year (an int) and \texttt{f} and \texttt{l} are the first and the last indices, respectively (both values are floats), for year \texttt{y}. 
    \item One tuple \texttt{(y, i)} in line for each year in the input file, where \texttt{y} is a year (an int) and \texttt{i} is the calculated inflation (a float), rounded to two decimal digits, given the first and the last index for year \texttt{y}.  
\end{enumerate}

