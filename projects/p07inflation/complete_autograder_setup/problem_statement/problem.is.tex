\problemname{Inflation}

Þetta verkefni er æfing í skráarvinnslu, listum og túplum.
Skrifið forrit sem les upplýsingar um vísitölu neysluverðs úr inntaksskrá og skrifar út ýmsar upplýsingar um gögnin.
Gögnin í inntaksskránum eru fengin af vef Hagstofu Íslands, \url{https://www.hagstofa.is/}.
Dæmigerð skrá (\texttt{data9091.txt}), sem inniheldur gögn fyrir árin 1990 og 1991, lítur svona út:

\begin{verbatim}
    1990M01	139.3
    1990M02	141.5
    1990M03	142.7
    1990M04	143.1
    1990M05	144.4
    1990M06	145.4
    1990M07	146.4
    1990M08	146.8
    1990M09	146.8
    1990M10	147.2
    1990M11	148.2
    1990M12	148.6
    1991M01	149.5
    1991M02	150.0
    1991M03	150.3
    1991M04	151.0
    1991M05	152.8
    1991M06	154.9
    1991M07	156.0
    1991M08	157.2
    1991M09	158.1
    1991M10	159.3
    1991M11	160.0
    1991M12	159.8    
\end{verbatim}
Í fyrsta dálkinum er ár/mánuður í hækkandi tímaröð og í seinni dálkinum er vísitala fyrir viðkomandi ár/mánuð.
Gögn úr inntaksskrá SKAL lesa inn í lista af túplum.

\section*{Inntak}
Inntakið er nafnið á skránni sem á að greina, t.d. \texttt{data9091.txt} eða hvaða annað nafn sem er sem endar á \texttt{.txt}.
Sérhver lína í inntaksskránni samanstendur af streng sem stendur fyrir ár/mánuð og rauntölu sem stendur fyrir vísitölu viðkomandi árs/mánaðar. 
Hvítt bil (e. whitespace) er á milli þessara tveggja sviða.

\section*{Úttak}
Þegar ekki er hægt að opna inntaksskrána er úttakið ekki neitt en annars samanstendur það af eftirfarandi upplýsingum:
\begin{enumerate}
    \item Ein túpla \texttt{(y, i)} í línu fyrir sérhverja línu í inntaksskránni, þar sem \texttt{y} er ár/mánuðir (strengur) og \texttt{i} er vísitala (rauntala).
    \item Ein túpla \texttt{(y, f, l)} í línu fyrir sérhvert ár í inntaksskránni, þar sem \texttt{y} er ár (heiltala), \texttt{f} er fyrsta vísitölugildi árs \texttt{y} og \texttt{l} er síðasta vísitölugildið (bæði vísitölugildin rauntölur).
    \item Ein túpla \texttt{(y, i)} í línu fyrir sérhvert ár í inntaksskránni, þar sem \texttt{y} er ár (heiltala) og \texttt{i} er útreiknuð verðbólga (rauntala), námunduð með tveimur aukastöfum, miðað við fyrsta og síðasta vísitölugildi árs \texttt{y}.  
\end{enumerate}

