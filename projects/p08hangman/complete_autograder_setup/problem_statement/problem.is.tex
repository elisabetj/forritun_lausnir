\problemname{Hangman}

Skrifið forrit sem gerir notanda kleift að spila leikinn Hangman.
Notandinn slær inn nafn skráar, sem inniheldur 20 ensk orð, og heiltölu sem er notuð til að velja tilheyrandi orð úr skránni sem leyniorðið.
Síðan reynir notandinn að finna leyniorðið með því að giska endurtekið upp á einstaka stafi. 
Forritið biður um stafainntak þangað til búið er að giska á alla stafi leyniorðsins eða að notandinn hefur notað 12 tilraunir.

Dæmigerð inntaksskrá (\texttt{data1.txt}) lítur svona út: 

\begin{verbatim}
lion
umbrella
window
computer
glass
juice
chair
desktop
laptop
dog
cat
lemon
cabel
mirror
hat
golf
chess
car
conference
test   
\end{verbatim}


\section*{Inntak}
Fyrsta lína inntaksins inniheldur nafn skráar sem geymir 20 ensk nafnorð í lágstöfum. 
Lengd sérhvers orðs er $\le 15$. 
Önnur lína inntaksins inniheldur heiltölu $i$, $1 \le i \le 20$, sem forritið notar til að velja orð nr. $i$ sem leyniorðið.
Sérhver lína nr. $n > 2$ inniheldur einn staf sem er ágiskun notandans í ítrun $n-2$.

\section*{Úttak}
Þegar ekki er hægt að opna inntaksskrána þá skrifar forritið ekkert út.
Annars, í sérhverri ítrun, er úttakið:
\begin{itemize}
    \item \texttt{"Secret word: - - - \ldots"} (eitt bandstrik eða réttur stafur fyrir sérhvern staf í leyniorðinu)
    \item \texttt{"Guess \{i\}/12"} (þar sem $1 \le i \le 12$)
    \item \texttt{"Correct letter \{c\}!"} eða \texttt{"Incorrect letter \{c\}!"} (þar sem \texttt{c} táknar stafinn í síðustu ágiskun)
\end{itemize}

Að lokum er síðasta línan í úttakinu: 
\begin{itemize}
    \item \texttt{"You won!"} eða
    \item \texttt{"You lost! The secret word was - - - \ldots"} (ef notandinn giskar 12 sinnum án þess að finna leyniorðið)
\end{itemize}
