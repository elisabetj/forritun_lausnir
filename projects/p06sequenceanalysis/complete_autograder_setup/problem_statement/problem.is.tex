\problemname{Sequence Analysis}

Þetta verkefni er æfing í skráarvinnslu, listum og frábrigðum.
Skrifið forrit sem les runu af rauntölum úr inntaksskrá og skrifar út ýmsar upplýsingar um rununa.
Dæmigerð runa í skrá (\texttt{data1.txt}) lítur svona út:

\begin{verbatim}
5
3
-2
7
-4
4
0
11
15
-12
22
6
8
\end{verbatim}

\noindent
En runa í skrá (\texttt{data2.txt}) gæti einnig verið svona:
\begin{verbatim}
12.4
11.9
-3.6
bla  
5.7
-2.6
7.4
8.9
xxx
4.5
\end{verbatim}

\noindent
Eða jafnvel svona (\texttt{data3.txt}):
\begin{verbatim}
aa
\end{verbatim}


Takið eftir að skráin \texttt{data2.txt} inniheldur tvær línur sem ætti ekki að taka með í greiningu á rununni, og skráin \texttt{data3.txt} inniheldur engar rauntölur.

\pagebreak

\section*{Inntak}
Inntakið er nafnið á skránni sem á að greina, t.d. \texttt{data1.txt} eða hvaða annað nafn sem er sem endar á \texttt{.txt}.
Inntaksskráin sjálf inniheldur $1-20$ línur, þar sem sérhver lína er annað hvort rauntala $f$, þar sem $-20 \le f \le 20$, eða einhver annar strengur sem er ekki leyfileg rauntala.

\section*{Úttak}
Úttakið er annað hvort ekki neitt, þegar inntaksskráin inniheldur engar rauntölur eða þegar ekki er hægt að opna inntaksskrána, eða samanstendur af eftirfarandi fjórum línum:
\begin{enumerate}
    \item Tölum í rununni
    \item Uppsafnaðri summu talnanna í rununni
    \item Raðaðri útgáfu (í hækkandi röð) af tölunum í rununni 
    \item Miðgildi rununnar
\end{enumerate}

Allar tölur á að námunda með fjórum aukastöfum. 

