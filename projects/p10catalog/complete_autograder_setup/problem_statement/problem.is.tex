\problemname{Catalog}

Í þessu verkefni eigið þið að útfæra tvo klasa: 
\texttt{Item} í skránni \texttt{item.py} og \texttt{Catalog} í skránni \texttt{catalog.py}.  
\begin{itemize}
    \item Smiðurinn (e. constructor) í \texttt{Item} tekur inn tvær færibreytur sem báðar eru strengir. Sú fyrri stendur fyrir nafn hlutarins og sú seinni fyrir flokkinn sem hluturinn tilheyrir.
    \item Smiðurinn í \texttt{Catalog} tekur inn eina færibreytu, streng sem stendur fyrir nafn listans (e. catalog).
    \item \texttt{Catalog} heldur utan um safn (e. collection) af \texttt{Item}. Safnið sem þið ákveðið að nota verður að viðhalda innsetningarröð (e. insertion order).
    \item Bæði \texttt{Item} og \texttt{Catalog} innihalda nokkrar tilvikaaðgerðir (e. instance methods). 
    Einstakar aðgerðir í klösunum \texttt{Catalog} og \texttt{Item} verða prófaðir með einingaprófum (e. unit tests). 
    Þið eigið að geta fundið út hvaða aðgerðir eru nauðsynlegar með því að skoða eftirfarandi aðalforrit og úttak þess:
\end{itemize}

\begin{verbatim}
from item import Item
from catalog import Catalog

item1 = Item("Green Book", "Biography")
print(item1)

item2 = Item("The God", "Crime")
print(item2)
item2.set_name("The Godfather")
print(item2)

item3 = Item("Schindler's List", "Biography")
print(item3)
item3.set_category("Drama")
print(item3)

catalog = Catalog("Films")
catalog.add(item1)
catalog.add(item2)
catalog.add(item3)
print(catalog)

catalog.remove(item2)
print(catalog)

catalog.set_name("Favorite Movies")
print(catalog)

catalog.clear()
print(catalog)
\end{verbatim}

\section*{Úttak}
Úttak aðalforritsins er eftirfarandi:
\begin{verbatim}
Name: Green Book, Category: Biography
Name: The God, Category: Crime
Name: The Godfather, Category: Crime
Name: Schindler's List, Category: Biography
Name: Schindler's List, Category: Drama
Catalog Films:
        Name: Green Book, Category: Biography
        Name: The Godfather, Category: Crime
        Name: Schindler's List, Category: Drama
Catalog Films:
        Name: Green Book, Category: Biography
        Name: Schindler's List, Category: Drama
Catalog Favorite Movies:
        Name: Green Book, Category: Biography
        Name: Schindler's List, Category: Drama
Catalog Favorite Movies:
\end{verbatim}

\textbf{Athugið:} Þegar einstök \texttt{Item} tiltekins \texttt{Catalog} eru prentuð út þá er einn tab-stafur í upphafi sérhverrar línu.
Aðalforritið á \textbf{ekki} að fylgja í skilunum ykkar, aðeins skrárnar \texttt{item.py} og \texttt{catalog.py}.