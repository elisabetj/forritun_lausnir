\problemname{Exam}

In this project, you implement a prototype of an exam system. 
The implementation consists of three classes: \texttt{Question} in the file \texttt{question.py}, \texttt{ChoiceQuestion} in the file \texttt{choice\_question.py}, and \texttt{Exam} in the file \texttt{exam.py}.

\section*{Question}
The Question class should have two private variables (and no other variables):
\texttt{\_\_question\_str} and \texttt{\_\_answer\_str}, denoting the question and the answer, respectively.
You should be able to figure out what methods you need by inspecting the following program, which tests the \texttt{Question} class, and the output examples below:

\begin{verbatim}
from question import Question

def answer_question(a_question):
    print(a_question.get_question())
    response = input()
    print(a_question.check_answer(response))

# Main
q = Question("Who is the inventor of Python?", "Guido van Rossum")
answer_question(q)
print(q)
\end{verbatim}

\noindent
Three output examples for the program above:

\begin{verbatim}
Who is the inventor of Python?
Guido van Rossum
True
Q: Who is the inventor of Python? A: Guido van Rossum

Who is the inventor of Python?
guido van rossum
True
Q: Who is the inventor of Python? A: Guido van Rossum

Who is the inventor of Python?
no idea
False
Q: Who is the inventor of Python? A: Guido van Rossum
\end{verbatim}

\noindent
\textbf{Note:} The result of calling \texttt{str()} is a string that contains both the question and the answer.

\section*{ChoiceQuestion}
This class extends the functionality of \texttt{Question} by providing a multiple choice question.
You should be able to figure out what variables and methods you need by inspecting the following program, which tests the \texttt{ChoiceQuestion} class, and the output examples below:

\begin{verbatim}
from choice_question import ChoiceQuestion

def answer_question(a_question):
    print(a_question.get_question())
    response = input()
    print(a_question.check_answer(response))

# Main
q = ChoiceQuestion("In what year was the Python language first released?")
q.add_choice("1991", True)
q.add_choice("1995", False)
q.add_choice("1998", False)
q.add_choice("2000", False)

answer_question(q)
print(q)
\end{verbatim}

\noindent
Two output examples for the program above:

\begin{verbatim}
In what year was the Python language first released?
1. 1991
2. 1995
3. 1998
4. 2000
1
True
Q: In what year was the Python language first released? A: 1

In what year was the Python language first released?
1. 1991
2. 1995
3. 1998
4. 2000
2
False
Q: In what year was the Python language first released? A: 1
\end{verbatim}

\noindent
\textbf{Note:} The correct answer to a choice question is a string corresponding to the number of the correct choice.
The number of choices for each choice questions are 2--4 (but you do not need to validate these counts).

\section*{Exam}
This class encapsulates an exam consisting of one or more instances of \texttt{Question} and/or one or more instances of \texttt{ChoiceQuestion}.  
You should be able to figure out what variables and methods you need by inspecting the following program, which tests the \texttt{Exam} class, and the output examples below:

\begin{verbatim}
from question import Question
from choice_question import ChoiceQuestion
from exam import Exam

q1 = Question("Who is the inventor of Python?", "Guido van Rossum")

q2 = ChoiceQuestion("In what year was the Python language first released?")
q2.add_choice("1991", True)
q2.add_choice("1995", False)
q2.add_choice("1998", False)
q2.add_choice("2000", False)

q3 = Question("What does OOP stand for?", "Object-oriented programming")

q4 = ChoiceQuestion("Which of the following is a built-in type in Pyhon?")
q4.add_choice("array", False)
q4.add_choice("record", False)
q4.add_choice("dict", True)

exam = Exam()
exam.add_question(q1)
exam.add_question(q2)
exam.add_question(q3)
exam.add_question(q4)

exam.take()
print(f"Your score is {exam.get_points()} point(s) out of {exam.get_num_questions()}")

print(exam)
\end{verbatim}

\noindent
An output example for the program above:

\begin{verbatim}
Who is the inventor of Python?
Not sure
False

In what year was the Python language first released?
1. 1991
2. 1995
3. 1998
4. 2000
1
True

What does OOP stand for?
object-oriented programming
True

Which of the following is a built-in type in Pyhon?
1. array
2. record
3. dict
2
False

Your score is 2 point(s) out of 4
Q: Who is the inventor of Python? A: Guido van Rossum
Q: In what year was the Python language first released? A: 1
Q: What does OOP stand for? A: Object-oriented programming
Q: Which of the following is a built-in type in Pyhon? A: 3
\end{verbatim}

\noindent
\textbf{Note:} When an exam is presented, questions are presented in the order they are added to the exam.
There is one empty line after the result (True or False) of each question has been printed out.

\section*{Instructions}
Make sure that you implement and test your classes (one at a time) in the order given above. 
You should only submit the files \texttt{question.py}, \texttt{choice\_question.py}, and \texttt{exam.py}.