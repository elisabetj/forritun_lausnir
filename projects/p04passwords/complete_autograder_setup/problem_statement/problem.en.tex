\problemname{Passwords}

Write a program which allows the user to repeatedly input a password until the input is \texttt{"q"}.
Each password $p$ should be error-checked according to the following conditions: 
\begin{enumerate} 
    \item $6 \le |p| \le 20$, where $|p|$ denotes the length of the password $p$.
    \item $p$ must contain at least one lowercase letter.
    \item $p$ must contain at least one uppercase letter.
    \item $p$ must contain at least one numeric letter.
\end{enumerate}

Conditions 2--4 should only be checked if the password fulfills condition no. 1. \\

In each iteration, an appropriate error message is printed depending on whether the password is valid or invalid.
Finally, the count of passwords and how many of them are valid/invalid is printed. \\

Note that the numbers 6 and 20, which stand for the minimum and maximum lengths of a password, as well as the string \texttt{"q"} (denoting quit), should be implemented with \textbf{constants}.
A constant in a Python program is a variable whose value should not be changed.
It is a good rule to use capital letters for names of constants and define them at the start of the program.

\section*{Input}
The input consists of $n$ lines, where $1 \le n \le 50$ and the last line is the string \texttt{"q"}.
Each line contains a string $s$, where $1 \le |s| \le 30$.
Each character in the input has an ASCII value between $48$ and $122$, inclusive.
This includes lowercase letters, uppercase letters, digits and some special symbols.

Note: Your program is not supposed to validate this input, or refuse other input.
This is just for your information about the input in the test cases. 
You do not need to expect input that does not meet these restrictions.

\section*{Output}
For each input line (except \texttt{"q"}), containing a password $p$, at most three lines are printed:
\begin{itemize}
    \item If $p$ is valid, the string "\{p\}: Valid password of len \{l\}." is printed and nothing else. 
    \item If $p$ is of invalid length, the string "\{p\}: Ìnvalid length." is printed and nothing else.
    \item If $p$ does not contain at least one lower case letter, the string "\{p\}: Missing lower case letter." is printed.
    \item If $p$ does not contain at least one upper case letter, the string "\{p\}: Missing upper case letter." is printed.
    \item If $p$ does not contain at least one numeric letter, the string "\{p\}: Missing numeric letter." is printed.
\end{itemize}
In the above, \{p\} is a placeholder for password $p$ and \{l\} is a placeholder for the length of $p$.
In the case that many lines are printed, their ordering should match the one in the list above.

Finally, the following string is printed:
"You tried \{c\} passwords, \{v\} valid, \{i\} invalid.", where \{c\}, \{v\}, and \{i\} are placeholders for the correct values.  

Note: You MUST use \textbf{f-strings/format strings} for all the output. 
