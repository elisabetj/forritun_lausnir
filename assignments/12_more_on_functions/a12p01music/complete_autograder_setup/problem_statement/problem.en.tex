\problemname{Music}

Write a function, \texttt{state\_music\_opinion(genre, music\_group, vocalist)},
that prints out certain music preferences, as detailed below.

\textbf{The main file, which handles input and output, is already provided. -
Please only submit your function definitions, without any code outside the functions!}


\section*{Input}

The function should accept three strings,
$g$, $b$ and $s$, in this order, as parameters,
but in case some of the arguments are not provided
the function should also default to the following values
for those arguments that are not provided:
``Classic Rock'' for $g$, ``The Beatles'' for $b$, and ``Freddie Mercury'' for $s$.

In the samples below,
the first and only line of the input contains
the three strings seperated by a ``\text{,}''.
The autograder will handle reading the input,
splitting it on the commas
and passing the resulting list as an argument to your function.
In case any of the strings are empty, or contain only whitespace,
those particular values will not be passed to your function,
in which case it should resort to the defaults.

Note that the parameters of the function
will sometimes need to be referred to by name,
so it is important that you use the following names
for the parameters in your function definition:
``\texttt{genre}'' for $g$, ``\texttt{music\_group}'' for $b$,
and ``\texttt{vocalist}'' for $s$.

In the tests, the strings $g$, $b$ and $s$
will consist of printable ASCII characters,
and have length at most $30$ each.


\section*{Output}

The function should not return anything (just \texttt{None}),
but should print out the following three lines:

\begin{itemize}
    \item ``\text{The best type of music is \{$g$\}.}''
    \item ``\text{The best music group is \{$b$\}.}''
    \item ``\text{The best lead vocalist is \{$s$\}.}''
\end{itemize}
