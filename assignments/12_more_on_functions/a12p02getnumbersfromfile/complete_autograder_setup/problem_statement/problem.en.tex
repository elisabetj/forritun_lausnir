\problemname{Get Numbers From File}

% \textbf{The main file, which handles input and output, is already provided. -
% Please only submit your function definitions, without any code outside the functions!}

Write a program that extracts all numbers from a given text file.
The results should be saved in a list and finally printed out in sorted order.

First, the program should ask the user for a file name,
repeatedly until the user enters a name of a file that actually exists,
as evidenced by the fact that it can be opened.


\section*{Input}

The input consists of one or more line.
At least one line will contain the name of a file
that is accessible to the program.

To be specific, the input will consist of $n$ lines, where $n \ge 1$.
For each $i$ with $1 \le i \le n$,
line $i$ will contain a non-empty string, let us call it $s_i$ for future reference.
At least one string $s_i$ will be the name of a file
that is accessible to the program.

Note that in the samples below, the content of the files is not shown.
See the provided text files for more information.
The input files will consist of $0$ or more lines,
containing a sequence of words and numbers separated by spaces.
Each word will be made up of letters from the English alphabet, parentheses and punctuation,
but there will never be numerical digits mixed into the words.
Any numbers will always stand on their own,
separated from the rest of the text by spaces or new line characters.


\section*{Output}

The output should consist of $m$ lines, with $m \ge 1$, as described below.

For each $i$ with $1 \le i < m$,
line $i$ of the output should read:
\begin{itemize}
    \item
    ``\texttt{\{$s_i$\} not found! Please try again.}''
\end{itemize}

Finally, the last line should contain an ordered list
of all numbers found in the file $s_m$,
separated by commas and surrounded by square brackets.

As soon as the program receives a name of a file that it can open,
it should stop asking for input and process the file.
Note that this means $m$ could be less than $n$,
but it should always be the case that $m \le n$.
