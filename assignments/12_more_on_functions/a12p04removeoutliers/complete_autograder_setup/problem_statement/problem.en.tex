\problemname{Remove Outliers}


Write two functions for removing the minimum number and the maximum number from integer lists:

\begin{itemize}
    \item
    \texttt{without\_outliers(a\_list)}:\\
    Returns a new list.
    Does not modify the list passed as an argument.
    % without the minimum number and the maximum number
    % from \texttt{a\_list},
    % without modifying \texttt{a\_list}.

    \item
    \texttt{remove\_min\_and\_max(a\_list)}:\\
    Returns nothing,
    modifies the list passed as an argument.
    % Removes the minimum number and the maximum number
    % from \texttt{a\_list},
    % thus modifying \texttt{a\_list}.
\end{itemize}

\textbf{The main file, which handles input and output, is already provided. -
Please only submit your function definitions, without any code outside the functions!}


\section*{Input}

Each of the functions should accept one list $l$ as a parameter.
You may assume the elements of the list $l$ will be integers.

In the samples below, the input consists of one line
containing a sequence $x = (x_1, \dots, x_n)$ of $n$ numbers
separated by a space.
The autograder will handle reading the input,
splitting it, converting to integers
and passing the resulting list as an argument to the functions.

In the tests, $x$ will always satisfy $2 \le n \le 1\,000$
and $0 \le x_i < 10\,000$ for all $i$ with $1 \le i \le n$.

\section*{Output}

The function \texttt{without\_outliers()}
should return a new list $m$
containing all the numbers appearing in $l$
except two, the largest and the smallest,
in the same order as they appear in $l$,
and no other numbers.
After the function is called,
the original list $l$ should remain unchanged.

The function \texttt{remove\_min\_and\_max()}
should return nothing (just \texttt{None}),
but should modify the given list $l$,
so that after the function is called,
the list will have changed to a modified list $l'$
containing all the numbers appearing in $l$
except two, the largest and the smallest,
in the same order as they appear in $l$,
and no other numbers.

In each test case, the autograder will call both of the functions
with the given list as a parameter, in sequence,
and will verify the results as well as the effects on the given list,
and print the following four lines to the output,
as seen in the samples below:
\begin{itemize}
    \item
    ``\texttt{Original list before calling functions: \{$l$\}}''

    \item
    ``\texttt{Resulting list after extracting middle: \{$m$\}}''

    \item
    ``\texttt{Original list after extracting middle and before removing outliers: \{$l$\}}''

    \item
    ``\texttt{Original list after removing outliers: \{$l'$\}}''
\end{itemize}
