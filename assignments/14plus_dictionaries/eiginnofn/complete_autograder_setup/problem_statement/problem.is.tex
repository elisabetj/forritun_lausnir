\problemname{Eiginnöfn}
\illustration{0.3}{doorbell}{Mynd fengin af \href{https://flickr.com/photos/david_e_smith/3616672234/}{flickr.com}}

Flestir Íslendingar eru annaðhvort með eitt eða tvö eiginnöfn.
Þau sem eru með tvö eiginnöfn eru oftast ávörpuð einungis með fyrra eiginnafninu, en ekki seinna.
Foreldrar þeirra eiga samt til að bregðast harkalega við þessu.
Þegar foreldrar Arnars voru spurðir hvort Arnar væri heima var svarið sjaldan einfalt já eða nei.
Í staðin var svarað: ,,Nei, en Arnar Bjarni er heima.``

Nú er búið að setja upp nýja snjalldyrabjöllu á heimilinu þar sem gestir geta skrifað nafnið á manneskjunni sem verið er að heimsækja.
Það eru samt áhyggjur yfir því að þetta sé ópersónuleg lausn og er ekki vilji fyrir að missa sjarmann og skemmtunina sem fólk upplifir frá foreldrabröndurum.
Því geymir dyrabjallan öll eiginnöfn hvers íbúa en í heimsóknarvalmyndinni er aðeins hægt að skrifa inn eitt eiginnafn til að spyrja eftir íbúa.
Dyrabjallan svarar þá hvort að íbúinn sé heima eins og foreldrar Arnars gerðu.
Ef íbúinn er með eitt eiginnafn og er heima, þá er svarað játandi.
Ef íbúinn er með tvö eiginnöfn og er heima, þá er svarað neitandi, en leiðrétt að manneskja með báðum eiginnöfnum íbúans sé heima.
Ef íbúinn er ekki heima að þá er svarað neitandi.

\section*{Inntak}
Inntak hefst á línu með einni heiltölu $n$, fjöldi íbúa sem eru heima.
Næst koma $n$ línur þar sem hver lína er ýmist með eitt eða tvö eiginnöfn sem eru aðskilin með bili.

Svo kemur lína með einni heiltölu $m$, fjöldi fyrispurna til dyrabjöllunnar.
Næst koma $m$ línur þar sem hver lína inniheldur eitt eiginnafn sem þýðir að spurt sé hvort sá íbúi sé heima.

Gera má ráð fyrir að engir tveir íbúar eigi sama fyrra eiginnafn.
Sérhvert eiginnafn byrjar á stórum staf og næst fylgja litlir stafir.
Hver einasti stafur í eiginnöfnunum er í enska stafrófinu.
Eiginnöfn eru í mesta lagi $10$ stafir á lengd.

\section*{Úttak}
Svara skal hverri fyrirspurn á eftirfarandi máta í sömu röð og fyrirspurnir koma fyrir í inntaki.
Ef íbúinn er ekki heima skal skrifa út \texttt{Neibb}.
Ef íbúinn er heima og hefur ekki seinna eiginnafn skal svara \texttt{Jebb}.
Ef íbúinn er heima og hefur tvö eiginnöfn skal svara á forminu \texttt{Neibb en <bæði eiginnöfn> er heima}.

\section*{Stigagjöf}
\begin{tabular}{|l|l|l|}
\hline
Hópur & Stig & Takmarkanir \\ \hline
1     & 10   & Sérhver íbúi hefur eitt eiginnafn og $1 \leq n, m, \leq 100$ \\ \hline
2     & 10   & Sérhver íbúi hefur tvö eiginnöfn og $1 \leq n, m \leq 100$ \\ \hline
3     & 20   & $1 \leq n, m \leq 100$ \\ \hline
4     & 20   & Sérhver íbúi hefur eitt eiginnafn og $1 \leq n, m, \leq 10^5$ \\ \hline
5     & 20   & Sérhver íbúi hefur tvö eiginnöfn og $1 \leq n, m \leq 10^5$ \\ \hline
6     & 20   & $1 \leq n, m \leq 10^5$ \\ \hline
\end{tabular}

