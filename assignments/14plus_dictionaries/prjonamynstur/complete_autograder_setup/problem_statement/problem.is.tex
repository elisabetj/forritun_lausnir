\problemname{Prjónamynstur}
\illustration{0.3}{knitting}{Mynd fengin af \href{https://www.flickr.com/photos/30478819@N08/50598685068}{flickr.com}}

Heiðrún hefur eindæma ánægju af því að prjóna alls konar flíkur á fjölskyldumeðlimi sína.
Þegar nýtt barn bætist í fjölskylduna má búast við að kjóll, húfa eða peysa sé næst á dagskrá hjá henni.

Þegar Heiðrún prjónar nýtir hún sér oft uppskriftir, einnig kölluð prjónamynstur, sem má finna í bókum, tímaritum eða jafnvel á alnetinu.
Þessar uppskriftir eru sýndar á myndrænan máta með reitum á tvívíðu hnitakerfi.
Til eru margar tegundir af lykkjum sem breyta útliti klæðanna.
Því hefur hver reitur í uppskriftinni tákn til að gefa til kynna hvernig tegund af lykkju á að prjóna.
Þar sem tegundir lykkja eru mismunandi að þá nota þær mismikið garn þegar þær eru prjónaðar.
Garnið er mælt í millímetrum og er kostnaður hverrar tegundar af lykkju eftirfarandi:

\begin{tabular}{|l|l|l|}
    \hline
    Tegund af lykkju                              & Tákn                    & Garn í millímetrum \\ \hline
    Garðaprjón                          & \texttt{.}              & $20$ \\ \hline
    Gatalykkja                          & \texttt{O}              & $10$ \\ \hline
    Einni óprjónaðri steypt yfir slétta & \texttt{\textbackslash} & $25$ \\ \hline
    Tvær saman til hægri                & \texttt{/}              & $25$ \\ \hline
    Þrjár saman                         & \texttt{A}              & $35$ \\ \hline
    Óprjónuð                            & \texttt{\^{}}           & $5$  \\ \hline
    Brugðin                             & \texttt{v}              & $22$ \\ \hline
\end{tabular}

Nú er Heiðrún að spá hvort hún sé með nóg garn fyrir næsta verkefnið sitt.
Ef hún sýnir þér uppskriftina sem hún ætlar að fylgja, geturðu sagt henni hversu mikið garn hún þarf?

\section*{Inntak}
Fyrsta línan inniheldur tvær heiltölur $n$, fjöldi raða í uppskriftinni, og $m$, fjöldi dálka í uppskriftinni.
Næst fylgja $n$ línur, með $m$ táknum hver, þar sem hver lína táknar eina röð í uppskriftinni.
Gera má ráð fyrir að uppskriftin innihaldi einungis tákn úr töflunni að ofan.

\section*{Úttak}
Skrifaðu út eina heiltölu, hversu mikið garn þarf fyrir uppskriftina í millímetrum.

\section*{Stigagjöf}
\begin{tabular}{|l|l|l|}
\hline
Hópur & Stig & Takmarkanir \\ \hline
1     & 40   & $1 \leq n, m \leq 50$ \\ \hline
2     & 40   & $1 \leq n, m \leq 1\,000$ og uppskriftin inniheldur einungis garðalykkjur. \\ \hline
3     & 20   & $1 \leq n, m \leq 1\,000$ \\ \hline
\end{tabular}

