\problemname{XOR}

Write a program that:
\begin{enumerate}
    \item Prompts the user for a file name.
    \item Opens the file in binary mode, using the \texttt{"rb"} mode of the \texttt{open} function.
    \begin{itemize}
        \item If the file does not exist, then the program should print \texttt{"No file named \{file name\} could be found"}.
    \end{itemize}
    \item Calculates an XOR checksum by performing byte-wise XOR operations for every byte in the file.
    \item Prints out \texttt{"The checksum is <checksum>"} where \texttt{<checksum>} is a single byte in hexadecimal format (e.g. \texttt{xB2}).
\end{enumerate}

\textbf{Example:} 
Let's assume that the file \texttt{a.bin} contains three bytes:
\begin{itemize}
    \item \texttt{0x01}, \texttt{0x02} and \texttt{0x01}.
\end{itemize}
Then the XOR checksum is:
\begin{itemize}
    \item \texttt{0x01 \textasciicircum{} 0x02 \textasciicircum{} 0x01 = 0x02}.
\end{itemize}

\section*{Input}
Input consists of:
\begin{enumerate}
    \item A filename containing the binary file to be processed.
\end{enumerate}

\section*{Output}
Output consists of:
\begin{enumerate}
    \item The XOR checksum of the binary file, presented as a single byte in hexadecimal format.
\end{enumerate}
