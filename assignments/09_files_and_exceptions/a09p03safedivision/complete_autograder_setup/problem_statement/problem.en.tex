\problemname{Safe division}

Write a program that prompts for two floating point numbers.
Then call a function named \texttt{divide\_safe(num1\_str, num2\_str)},
which returns the result of dividing the first number by the second number.
If the given strings do not represent floating point numbers
or the second number is 0, the function returns \texttt{None}.
Use \texttt{try-except} and catch two types of exceptions:
\texttt{ValueError} and \texttt{ZeroDivisionError}.

The main program prints the result rounded to 5 decimal digits if no error occurred, else prints \texttt{None}.

\section*{Input}
The input consists of:
\begin{enumerate}
    \item A string $s_1$ representing the first number.
    \item A string $s_2$ representing the second number.
\end{enumerate}
Note that the strings may contain symbols which are not digits.
Each line will contain at most $30$ symbols.

\section*{Output}
The output consists of:
\begin{enumerate}
    \item The result of the division rounded to 5 decimal digits or \texttt{None} in case of an error.
\end{enumerate}
