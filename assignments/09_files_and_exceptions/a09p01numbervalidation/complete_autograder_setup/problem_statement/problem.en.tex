\problemname{Number Validator}

Write a function named \texttt{is\_float(string\_to\_test)}
that takes one argument that is a string.
It returns \texttt{True} if the string \texttt{string\_to\_test} represents a floating point value
and returns \texttt{False} otherwise.

You are required to use \texttt{try}-\texttt{except}.
The basic concept is to \emph{try} to convert \texttt{string\_to\_test} to a float
and if it succeeds, return \texttt{True},
but if it fails (that is, an exception is raised), return \texttt{False}.
Note that \texttt{float()} raises a \texttt{ValueError} exception.

You should probably not need to implement any helper functions for this task,
but if you wish, you are welcome to do so,
although then you are probably overcomplicating things,
as there is not much code required to solve this.

\section*{Submission instructions}

\textbf{Note that we are testing this function specifically, so your solution must define it.
Please only submit your function definition(s), without any code outside the function(s)!}

No code is necessary other than your function definition(s),
but your solution can contain other code if you can't be bothered to remove it,
as long as it is restricted to
\texttt{if \_\_name\_\_ == "\_\_main\_\_":}
so it will not run when Gradescope is importing your function.
The main python file, which handles input and output, is already included on Gradescope.

A similar, simpler, main file is provided on Canvas for your convenience, along with the starter code.
You can download and place that main file in the same directory as your python file,
and supply the name of your python file for importing the function \texttt{is\_float} in the main file.
You can then run the main python file we provide to try out the samples.
See also the given starter code for an alternative setup.

\section*{Input}

Your function should accept one parameter,
called \texttt{string\_to\_test},
expected to be of type \texttt{str}.
In this description, we will refer to the input string as $s$, for short.
Gradescope will handle reading the input for you.
The input will consist of a single line containing a string,
and that will be passed as an argument to your function.

For your information, in the test cases,
the input string $s$ will have at most $20$ symbols,
consisting of English letters, digits, spaces, and various symbols.
To be specific, each symbol has an ASCII value between $32$ and $126$, inclusive.
You do not need to validate the input, or refuse other input,
this is just to inform you that you do not need to worry about other kinds of input.

\section*{Output}

Your function should return a boolean value,
indicating if the string $s$ can be interpreted as a \texttt{float} or not.
Gradescope will handle printing the output for you.
The output should consist of one line with either a \texttt{True} or \texttt{False}.
