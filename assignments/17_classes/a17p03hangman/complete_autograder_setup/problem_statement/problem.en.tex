\problemname{Hangman}

Write the class \texttt{Hangman} which represents a word in the game hangman.
The class should keep track of which letters of the word have been revealed and how many times an incorrect guess has been made,
and it should have the following methods:
\begin{itemize}
    \item \texttt{\_\_init\_\_(word: str)}: Initializes the word to be discovered, as well as other attributes as necessary.
    \item \texttt{guess\_letter(letter: str) -> bool}: Returns whether the guess was correct.
        If the guess was correct, the letter should now no longer be hidden.
    \item \texttt{\_\_str\_\_() -> str}: Shows the word, only displaying the revealed letters, and the number of incorrect guesses. 
        For those that are still hidden, show ``\_'' instead. Letters should be printed in uppercase.
\end{itemize}

It is a good idea to break the \texttt{guess\_letter()} method down by delegating some of the details to helper functions.
Make sure to have methods call other methods where applicaple, to avoid implementing the same functionality in multiple places.

\textbf{Note:} You should \textbf{not} submit the main program as part of your solution, only the file that contains the class \texttt{Hangman}.

The following is an example main program, this example is also included in the supplied ``\texttt{main.py}'' file:
\begin{verbatim}
hangword = Hangman("Testing")
print(hangword)
print(hangword.guess_letter("i"))
print(hangword.guess_letter("I"))
print(hangword.guess_letter("a"))
hangword.guess_letter("t")
print(hangword)
hangword.guess_letter("E")
hangword.guess_letter("s")
print(hangword)
hangword.guess_letter("i")
hangword.guess_letter("N")
hangword.guess_letter("g")
print(hangword)
\end{verbatim}

\section*{Output}
the corresponding outputs for the sample program given above:

\begin{verbatim}
_ _ _ _ _ _ _
Number of incorrect guesses: 0
True
True
False
T _ _ T I _ _
Number of incorrect guesses: 1
T E S T I _ _
Number of incorrect guesses: 1
T E S T I N G
Number of incorrect guesses: 1
\end{verbatim}
