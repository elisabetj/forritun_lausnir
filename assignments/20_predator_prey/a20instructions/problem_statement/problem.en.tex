\problemname{Predator Prey}

In this project, you are given an implementation of the \texttt{Predator-Prey} project from chapter $13$ in the textbook.
The implementation, which can be found on Canvas,
consists of the following files: 
\begin{itemize}
    \item``\texttt{main.py}'', 
    \item``\texttt{island.py}'', 
    \item``\texttt{animal.py}'', 
    \item``\texttt{predator.py}'', 
    \item``\texttt{prey.py}''.
\end{itemize}

Along with these files each part will have their own sample input and output files also provided on Canvas.

You should have prepared for this class by reading chapter $13$ and studying the solution given there.
Thus, you should be familiar with both the problem and the main parts of the solution.

The implementation we provide is based on the one in the textbook, but with some minor changes.
In addition to the minor changes, the provided solution contains the ``move refinement'' discussed in chapter $13.5$ using ``marking''.

Your task in this project is twofold.
First, to find errors in the given implementation and correct them, this is an exercise in using a debugger.
Second, to implement added functionality.

\section*{Input}

For all three parts of this problem the input will consist of one line 
containing one integer, the random seed which the game will be played on.

\section*{Output}

For all three parts the output will consist of the state of the game for each
simulated game tick.

\textbf{Note:} you should not have to change any of the input nor output in this assignment
this section is strictly to help you grasp the inputs and outputs better.
