\problemname{Trials and Triangulations}

More math? You got it. That's what computers do, they compute.

Heron's formula gives the area, $A$, of a triangle with sides $a$, $b$, and $c$ as $A = \sqrt{s(s-a)(s-b)(s-c)}$
where $s = (a+b+c)/2$.

Write a program that prompts for three integers, $a$, $b$, and $c$, denoting the lengths of the sides of a triangle.
Calculate the area, and print the result.

Hint: You can use the \texttt{sqrt} function in the \texttt{math} module.

\section*{Input}
Input consists of three lines.
The first line consists of one integer $a$.
The second line consists of one integer $b$.
The third line consists of one integer $c$.
It is guaranteed that $1 \leq a, b, c \leq 100$ and that the values can form a triangle.

\section*{Output}
Output one line with one floating point number $A$, the area of the triangle.
The output number should have an absolute or relative error of at most $10^{-9}$.
