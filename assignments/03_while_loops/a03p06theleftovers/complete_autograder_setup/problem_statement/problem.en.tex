\problemname{The Leftovers}

The game \emph{Rock, paper, scissors} is well known,
and often used when two people need to choose one among them
for some activity.

But what if there are more than two people who need to select a candidate?

Then the game can be changed/extended as follows
(and this version can be used for two people as well):

The players are numbered from $0$ to $n-1$,
where $n$ is the number of players.

On the count of three, each person puts out a number of fingers.
All upheld fingers are counted,
and the total is divided by the number of players.
The remainder of that division,
which will be a number in the range $0$ to $n-1$ (inclusive),
decides which player is chosen.

Write a program that:
\begin{enumerate}

\item First, asks how many players are involved.
Let us keep it to two or more.

If the given input is less than $2$,
your program should keep asking until the input is valid.

\item Next, asks each player about their contribution.
In the physical game,
the contribution would be limited by the number of fingers,
but in a computer implementation we do not have such primitive limitations.

\item Finally, sums up all contributions and calculates who is the winner.

\end{enumerate}

\section*{Input}
Input consists of one integer $p$, the number of players, where $ 2 \leq p \leq 100$. If $p < 2$ this input is repeated.
This is followed by $p$ lines of one integer $c$ each, the contribution of each player, where $ -100\,000 \leq c \leq 100\,000$.

\section*{Output}
Output consists of three lines.\\
The first line contains the string \texttt{The sum of all contributions is (t)}\\
The second line contains the string \texttt{When (t) is divided by (p), the remainder is (r)}\\
The third line contains the string \texttt{Player (r) is the winner!}\\
Where $t$ is the total contributions, $r$ is the number of the player that won, and $p$ is the number of players.

