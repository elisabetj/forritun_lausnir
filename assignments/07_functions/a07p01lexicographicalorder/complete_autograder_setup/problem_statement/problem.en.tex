\problemname{Lexicographical Order}

Write a function, \texttt{precedes(one\_string, another\_string)}, that receives two strings as parameters.
The function returns the string that comes first in alphabetical order.
For example, Hungry comes before Starving alphabetically, because h comes before s,
and Duchess precedes Duke, because both strings start with du, but c precedes k.
The function should ignore case.

\textbf{Note that we are testing your code differently in this task,
please only submit your function definitions, without any code outside the functions!}
The main python file, which handles input and output, is already provided.
You can download and place the main file in the same directory as your python file.
You can then run the main python file we provide to try out the samples.

\section*{Input}
Your function will be called with two given parameters as input.
Each parameter will be a string, consisting of $1$ to $20$ letters from the English alphabet and spaces.
You do not need to check this.

As usual, the input is given in an input file, and a description of that follows below.
But the given main file will take care of reading the input and passing the parameters to your function,
as well as printing the result to the output for you, so you don't really need to worry about that,
except for when you want to test your code locally, then you might want to do something similar.
The samples given below show the input and corresponding output as they appear in these files.

The input consists of two lines.
Each line contains one string,
and those are the ones that will be passed as the parameters to the function.

Formally (since we know that mathematical notation is your favourite language),
the input consists of $n = 2$ lines,
and line $i$ contains a string $s_i$, for $1 \le i \le n$.
Further, $1 \le |s_i| \le 20$, where $|s|$ denotes the length of a string $s$.

\section*{Output}
The output of the function should be a string, the same string as one of the input strings,
the one that precedes the other alphabetically.
The comparison should be made ignoring case,
but the result should be the original string unchanged.

This will result in an output file that is one line containing the value returned from calling \texttt{precedes},
with the first input string $s_1$ passed as the first argument to the function,
and the second input string $s_2$ passed as the second argument.

The function \texttt{precedes} will be called exactly once,
in each execution of the program, which is to say for each test case.
