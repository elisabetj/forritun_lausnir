\problemname{Email Addresses}

\illustration{0.4}{mail}{Mail}

You have started your own company and are creating
a \texttt{Software-as-a-Service} solution for tracking missing socks,
called \texttt{where-my-sock.is}.
You are currently in a beta phase.
People who are interested in using the product
need to sign up with an email address in order to receive an invite.

You have set up an error monitoring system so that you are notified whenever an error occurs in your software system.
Recently, you have noticed several errors from the email system that sends out the invites.
You investigate and discover that a surprisingly large portion of people trying to sign up
misspell their email address when they fill out the signup form.
The service you use to send the email messages is very picky about email addresses
\href{https://en.wikipedia.org/wiki/Email\_address}{conforming to specification}.
This is a bit of a shame because these users are not made aware of their mistake
and instead end up refreshing their inboxes, never receiving the invite they crave so much.

You decide to do something about this.\\
You look at the error logs and compile a list of the $8$ most common mistakes people make.
They are:
    \begin{enumerate}
    \item $@$ symbol is missing.
    \item There are multiple $@$ symbols.
    \item There is nothing before the $@$ symbol.
    \item There is nothing after the $@$ symbol.
    \item Email address starts with a dot.
    \item There is a dot just before the $@$ symbol.
    \item There are consecutive dots.
    \item The top-level-domain (e.g. \texttt{.com}) is missing.
    \end{enumerate}

You intend to add a validator to the signup form
to provide a helpful message to users when they make one of these mistakes,
but first you create a prototype in your favourite programming language Python.

\section*{Input}
Input consists of one line that includes the email address that should be checked.

\section*{Output}
Output consists of an explanation of what is wrong with the email according to the rules above
or the phrase \texttt{All good.} if nothing is wrong.
The rules should be validated in the same order as stated above.

\section*{Explanation of samples}
Sample outputs $1-8$ correspond to the expected output for the most common mistakes $1-8$.
