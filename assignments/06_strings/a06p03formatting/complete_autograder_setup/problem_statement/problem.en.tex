\problemname{Formatting}

A floating point number is a rational number expressed as a decimal. In other words, a number made up of digits and at most one decimal point. \\
Write a program that, given a floating point number, formats it such that it follows the following specifications:

\begin{enumerate}
	\item Has a field width of 12
	\item Has exactly 2 decimal digits of precision
	\item Is right justified
\end{enumerate}

\section*{Input}
Input consists of one line containing one floating point number $f$, with at most $18$ digits after the decimal point, where $0 \leq f \leq 10^9$ \\

\section*{Output}
Output consists of a single line, the formatted floating point number.
Note that your output must be exactly as described above.
This means it must have the correct number of leading spaces and finally one newline character at the end of the number.
