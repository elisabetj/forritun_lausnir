\problemname{Countries}

Write a program that reads from a text file
a list of countries in the world,
and then lists all country names of a given length.

The name of the text file is supplied by the user,
so the program should begin by collecting the file name.
In case the given file name is invalid,
the program should report the fact, and demand a new one,
repeatedly until a valid file name is entered.

Once a valid file has been given,
the program should allow the user to repeatedly supply an integer,
and each time the program should respond by listing all countries
the length of whose names is equal to that integer.

Rather than having to go through the entire list
each time the user queries for a specific length,
the program should process the country list
before starting to take length queries from the user.

That's where the subject of todays class comes in, dictionaries.
As the program reads the contents of the given file,
it should check the length of each country name,
and organize the countries into a dictionary,
where countries with equally long names are grouped together.

So the program should create a dictionary where the key is an integer
representing the length of a country name
and the value is a list of country names of that length.
For example, an entry in the dictionary might look like this:
\begin{itemize}
    \item
    \texttt{\{16 : ["Marshall Islands", "Papua New Guinea"]\}}
\end{itemize}

Note: the file \texttt{countries.txt} that is used in the samples
is attached for your convenience.


\section*{Input}

% First, a general description of the input,
% explaining to students how to interpret
% what they are seeing in the samples below.
The first line of the input is taken to be the name of a file,
and possibly more lines as well,
as many as needed until a valid file name is provided.
After that, the rest of the input will come in pairs,
an integer on one line, denoting a query,
followed by a line containing a string
indicating whether there are more queries to come,
until the string will be the single letter \texttt{n}.
Each line in the input file will contain one country name.

\subsection*{Formal input specifications}
% Then a more precise description, leaving nothing to doubt.
Formally, the input will consist of $m + 2n$ lines,
$l_1, \dots, l_{m+2n}$,
where $m$ denotes the number of file name attempts, with $1 \le m$,
and $n$ denotes the number of queries, with $1 \le n$.

For $i < m$, line $l_i$ will contain an invalid file name attempt $a_i$,
and line $l_m$ itself will contain a valid file name $a_m$.

For $j \in \{1, \dots, n\}$,
line $l_{m + 2j - 1}$ will contain an integer $q_j$,
and line $l_{m + 2j}$ will contain a string $r_j$,
denoting a reply to the question whether more queries will follow.
The reply $r_n$ will consist of the string \texttt{n},
indicating that no more queries are forthcoming.
All other replies $r_j$ for $j < n$
will be something other than \texttt{n}.

So line $l_{m + 1}$ will contain a query $q_1$,
line $l_{m + 2}$ a reply $r_1$,
line $l_{m + 3}$ a query $q_2$ and so on,
until the reply $r_n$ will be \texttt{n}.

The input file will consist of $k$ lines, with $0 \le k$,
and each line will contain one country name, $c_i$,
for $i \in \{1, \dots, k\}$.

\subsection*{Test case restrictions}
% And finally, assurances about what to expect from the test cases.
% Here we mention what we will restrict the input to in the tests,
% to clarify that they do not need to worry about input outside those restrictions.
% This does not mean their solutions should validate the input,
% or reject input that does not satisfy those restrictions.
In the tests, $m$ and $n$ will be restricted to
$m \le 10$ and $n \le 1\,000\,000$.
The length of each file name attempt $a_i$
will be restricted to $1 \le |a_i| \le 100$.
Each query $q_j$ will be restricted to $1 \le q_i \le 50$,
and each reply $r_j$ will be either \texttt{y} or \texttt{n},
with $r_j$ being \texttt{y} for all $j < n$,
and the last reply $r_n$ being \texttt{n}.

As for the contents of a valid input file,
$k$ will be restricted to $0 \le k \le 1\,000$,
and the length of each country name $c_i$
will be restricted to $1 \le |c_i| \le 40$.
The countries will appear in alphabetical order in the file.


\pagebreak
\section*{Output}

% First, a general description of the output,
% explaining to students how to interpret
% what they are seeing in the samples below.
For each invalid file name attempt $a_i$, with $i < m$,
the program should report that the file was not found.
And then, for each query $q_j$, with $j \in \{1, \dots, n\}$,
the program should display all countries
whose names have length $q_j$,
on a single line separated by commas,
or state that no such country was found.

\subsection*{Formal output specifications}
% Then a more precise description, leaving nothing to doubt.
So, the output should consist of $m-1 + n$ lines in total,
call them $\lambda_1, \dots, \lambda_{m-1 + n}$.
For $i < m$, line $\lambda_i$ should contain the string:
\begin{itemize}
    \item
    \texttt{File \{$a_i$\} not found!}
\end{itemize}
And then, for each query $q_j$, with $j \in \{1, \dots, n\}$,
line $\lambda_{m-1 + j}$ should contain
all country names $c_i$ satisfying $|c_i| = q_j$,
separated by commas as well as a space,
as long as there is at least one such country.
For example, if there are $3$ country names
$c_{i_1}, c_{i_2}, c_{i_3}$ of length $q_j$,
then line $\lambda_{m-1 + j}$ should be:
\begin{itemize}
    \item
    \texttt{\{$c_{i_1}$\}, \{$c_{i_2}$\}, \{$c_{i_3}$\}}
\end{itemize}
% The country names should be listed in alphabetical order
The country names should be listed in the order they appear in the file.
% (you may assume that they will be arranged alphabetically in the input file).
% Apparently the text file is not sorted alphabetically as I thought.
% TODO: Fix this.

If there are no countries of the given length $q_j$,
then line $\lambda_{m-1 + j}$ should contain:
\begin{itemize}
    \item
    \texttt{No country name of length \{$q_j$\} exists.}
\end{itemize}
