\problemname{Gradebook}

Write a program that repeatedly asks for a student email and a grade,
and when all grades have been entered,
possibly multiple grades for the same email,
the program calculates the average grade corresponding to each email.

You should use a dictionary to keep track of the grades corresponding to each email,
keeping a list of grades as a value for each email that is given,
so if the same email is entered twice, the new grade should be appended to the list.


\section*{Input}

% First, a general description of the input,
% explaining to students how to interpret
% what they are seeing in the samples below.
The input will come in triples,
where in each triple,
the first line will be an email address,
the second line will be a grade
and the third line will be a response
indicating whether another email-grade pair will be entered.

\subsection*{Formal input specifications}
% Then a more precise description, leaving nothing to doubt.
Formally, the input will consist of $3n$ lines,
$l_1, \dots, l_{3n}$,
where $n$ is the number of grades that will be entered, with $1 \le n$.
For each $i \in \{1, \dots, n\}$,
line $l_{3i - 2}$ will contain a string containing an email address $e_i$,
line $l_{3i - 1}$ will contain an integer denoting a grade $g_i$,
and line $l_{3i}$ will contain a string giving a reply $r_i$
to the question whether more grades will follow.
The reply $r_n$ will consist of the string \texttt{n},
indicating that no more grades are forthcoming.
All other replies $r_j$ for $j < n$
will be something other than \texttt{n}.

So line $l_1$ will contain an email $e_1$,
line $l_2$ a grade $g_1$,
line $l_3$ a reply $r_1$,
line $l_4$ an email $e_2$
and so on,
until the reply $r_n$ will be \texttt{n}.

\subsection*{Test case restrictions}
% And finally, assurances about what to expect from the test cases.
% Here we mention what we will restrict the input to in the tests,
% to clarify that they do not need to worry about input outside those restrictions.
% This does not mean their solutions should validate the input,
% or reject input that does not satisfy those restrictions.
In the tests, $n$ will be restricted to
$n \le 1\,000\,000$.
The length of each email address $e_i$ will be restricted to
$5 \le |e_i| \le 20$.
Each grade $q_j$ will be an integer restricted to $0 \le g_i \le 10$,
and each reply $r_j$ will be either \texttt{y} or \texttt{n},
with $r_j$ being \texttt{y} for all $j < n$,
and the last reply $r_n$ being \texttt{n}.


\section*{Output}

% First, a general description of the output,
% explaining to students how to interpret
% what they are seeing in the samples below.
The program should display each email that was entered,
along with the average grade corresponding to the email,
the emails should be sorted in alphabetical order.
The grade should be rounded to 2 decimal places.

\subsection*{Formal output specifications}
% Then a more precise description, leaving nothing to doubt.
Let $m$ be the number of distinct emails in the list
$e_1, \dots, e_n$,
and let $u_1, \dots, u_m$
be a list of the $m$ unique emails, ordered alphabetically.
For each $j \in \{1, \dots, m\}$,
associate a grade $g_i$ with $u_j$ if $e_i = u_j$,
and let $a_j$ be the average value of the grades associated with $u_j$.

Then the output should consist of $m$ lines,
$\lambda_1, \dots, \lambda_m$
and for each $j \in \{1, \dots, m\}$,
line $\lambda_j$ should contain
the email $u_j$ and the average grade $a_j$,
separated by a colon and a space:
\begin{itemize}
    \item
    \texttt{\{$u_j$\}: \{$a_j$\}}
\end{itemize}
