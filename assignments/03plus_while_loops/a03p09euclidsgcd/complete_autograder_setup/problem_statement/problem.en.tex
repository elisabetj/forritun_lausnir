\problemname{Euclid's GCD}

One of the oldest known algorithms
is a method for finding the greatest common divisor of two natural numbers.
It turns out that you have already learned all you need to implement it.

Given two numbers $a$ and $b$,
we keep doing the following until $b$ becomes $0$:
\begin{itemize}
    \item Calculate the remainder of dividing $a$ by $b$,
    \item update $a$ to be the previous value of $b$ and $b$ to be the value of the remainder.
\end{itemize}
If at any time $b$ is $0$, then $a$ is the greatest common divisor of the original two numbers.

\section*{Input}
Input consists of two lines.
The first line consists of one integer $a$, where $0 \leq a \leq 10^{18}$.
The second line consists of one integer $b$, where $0 \leq b \leq 10^{18}$.

\section*{Output}
Output the greatest common divisor of $a$ and $b$.
