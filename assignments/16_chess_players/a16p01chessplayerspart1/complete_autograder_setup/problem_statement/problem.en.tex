\problemname{Chess: Part I}

In this first part, the purpose of the program is to display another view of the data,
by grouping the data by country.

In addition to the above requirement,
you should make sure that your program can easily look up a given chess player
and display their attributes (rank, country, rating, birth year),
even though that functionality is not part of the current requirements.


\section*{Input}

The input consists of a single line,
containing the name of an input file
which will be made accessible to the program.


\section*{Output}

The output should contain
a table of player rankings grouped by country,
the beginning of which should look like the following:

% TODO: trim output
\begin{verbatim}
Players by country:
-------------------
ARG (1) (2653.0):
                           Sandro Mareco      2653
ARM (3) (2710.3):
                           Levon Aronian      2780
                      Gabriel Sargissian      2691
                         Hrant Melkumyan      2660
AUT (1) (2690.0):
                           Markus Ragger      2690
AZE (6) (2722.3):
                   Shakhriyar Mamedyarov      2820
                        Teimour Radjabov      2751
                       Arkadij Naiditsch      2721
                            Rauf Mamedov      2699
                           Eltaj Safarli      2676
                          Gadir Guseinov      2667
BLR (1) (2664.0):
                       Vladislav Kovalev      2664
CHN (9) (2718.7):
                              Liren Ding      2804
                               Yangyi Yu      2765
                                  Yi Wei      2742
                                Hao Wang      2722
                             Xiangzhi Bu      2712
                               Chao b Li      2708
                                Yue Wang      2681
                                  Hua Ni      2676
                               Yifan Hou      2658
\end{verbatim}

So, the first two lines should contain the table header, on the first line,
accompanied by an equally long line of dashes on the second line.
After that, each country that has at least one player among the 100 top scoring players,
should make an appearance in alphabetical order,
as a subheader,
followed by a list of the players from that country.

The numbers in parenthesis are
the number of players in each group and
the average chess ratings of those players.

For each country, the players should be listed in order of rating,
the highest scoring first, and so on down the list.
As it happens, this is the same order as they appear in the input file,
so you do not need to do anything to sort them.

When printing the chess player name and rating, use this format:
\texttt{\{:>40\}\{:>10\}}
