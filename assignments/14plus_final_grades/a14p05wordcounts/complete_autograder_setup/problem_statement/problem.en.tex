\problemname{Word counts}

Write a program that reads a text file,
keeps a count of the individual words in the file using a dictionary
and then groups the words by how often they appear,
and prints the words of each count, most frequent first.
The main program is given in the starter code, do not change it.
Implement the remaining given functions.

Note that the counts are not case-sensitive, that is, 'Word' is the same as 'word' or 'wORd'.

Also, note that your program should account for if a punctuation, like a comma, appears at the end of a word,
meaning that words like ``university'' and ``university,'' are the same word.

The input files that are used in the samples,
are provided for your convenience.


\section*{Input}

% First, a general description of the input,
% explaining to students how to interpret
% what they are seeing in the samples below.
A line in the input specifies the name of a file to be processed.
There might be multiple lines, and some lines may contain invalid file names,
but at least one of them will be valid.

% \subsection*{Formal input specifications}
% Then a more precise description, leaving nothing to doubt.
Formally, the input consists of $n$ lines,
$l_1, \dots, l_n$, where $1 \le n$,
and at least one of them will contain the name of a file accessible to the program.
Let us refer to the first such file as $f$.

The file $f$ will consist of $k$ lines,
containing a total of $m$ words,
$w_1, \dots, w_m$,
where $1 \le k$ and $1 \le m$.

% \subsection*{Test case restrictions}
% And finally, assurances about what to expect from the test cases.
% Here we mention what we will restrict the input to in the tests,
% to clarify that they do not need to worry about input outside those restrictions.
% This does not mean their solutions should validate the input,
% or reject input that does not satisfy those restrictions.
In the tests, $n$, $k$ and $m$ will be restricted to
$n \le 100$, $k \le 1\,000\,000$ and $m \le 1\,000\,000\,000$.


\section*{Output}

% First, a general description of the output,
% explaining to students how to interpret
% what they are seeing in the samples below.
The exact output specification is tedious to communicate,
but it can be well understood by looking at the samples.

% \subsection*{Formal output specifications}
% Then a more precise description, leaving nothing to doubt.
