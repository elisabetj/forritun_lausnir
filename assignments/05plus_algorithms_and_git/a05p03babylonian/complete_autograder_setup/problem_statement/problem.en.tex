\problemname{Babylonian Square Root Algorithm}
This exercise is based on chapter $3.5$ in the textbook.

Let us implement the Babylonian square root algorithm.
Here is a pseudocode description of the algorithm:
    \begin{itemize}
        \item Let $n$ be the number for which to find the square root.
        \item Let $g$ be the initial guess.
        \item Let $t$ be the tolerance.
        \item Let $g^\prime$ be our previous guess, initialized to $0$.
        \item While the absolute difference between $g$ and $g^\prime$ is greater than $t$:
            \begin{itemize}
                \item $g^\prime = g$.
                \item $g =\frac{\frac{n}{g} + g}{2}$, the average of $\frac{n}{g}$ and $g$.
            \end{itemize}
    \end{itemize}

\section*{Input}
Input consists of three lines and they are as follows:

    \begin{itemize}
        \item $n$ - an integer for which to find the square root, where $1 \leq n \leq 100\,000$.
        \item $g$ - an integer representing from where to start the search for the actual square root, where $1 \leq g \leq 100\,000$.
        \item $t$ - a floating point number, indicating the minimum change between iterations until the algorithm stops, where $0.1 \leq t \leq 0.000\,01$. The number is given with at most $5$ digits after the decimal point.
    \end{itemize}

\section*{Output}
Output consists of two lines.
The first line contains the square root of $n$,
and the second line contains the number of repetitions until the tolerance was reached.
The output of the square root should have an absolute or relative error of at most $10^{-4}$.
