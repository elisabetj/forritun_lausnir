\problemname{Insert}

Now write the function
\texttt{insert\_at(sequence, index, element)}
mentioned previously.

\textbf{Note that we are testing your code differently in this task,
please only submit your function definitions, without any code outside the functions!}
A skeleton of a main python file, which handles input and output, is provided,
but by now you should have a good idea of how to set up a main file to test running your code.

\section*{Input}
The function receives three parameters,
a sequence $s$, an integer $i$ and an element $e$.
In the tests, $s$ will be a string with $0 \le |s| \le 19$,
$e$ will be a character, that is, a string of length $|e| = 1$,
and $i$ will be restricted to $0 \le i \le |s|$.

It is good if your function also works for other types of sequences and elements,
or for input outside these specifications,
but that is not part of the requirements.

In the samples below,
the first line of the input contains the string $s$,
the second line contains the index $i$
and the third line contains the character $e$.

\section*{Output}

The function should return a sequence $s'$,
identical to the input sequence $s$
except with the element $e$ inserted at index $i$,
and the elements behind that index shifted to the right,
so $|s'| = |s| + 1$.
In particular, given a string $s$ as input, $s'$ should also be a string.

Note that $i$ can be equal to $|s|$,
which means $e$ should be appended to the right end of $s$.
In fact, $i$ will end up being an index pointing to a position in $s'$,
the position where $e$ should be after insertion.

In the samples below,
the first and only line of the output contains the sequence $s'$.
