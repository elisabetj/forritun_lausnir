\problemname{Tetris!}

\illustration{0.4}{tetris}{Tetris}

You make a rather profound realization whilst taking a shower.
Using your two functions, \texttt{rotate\_text\_clockwise()} and \texttt{change\_indentation()}, 
you should be able to make a crude ASCII-art version of Tetris!

Well, at least you could ``draw'' one block, move it left or right and rotate it.
You rush back to your desk to hack together a prototype.

You decide to start by ``drawing'' one piece using Xs like so:
\begin{verbatim}
X
XXX
\end{verbatim}

You then add a loop that repeatedly asks the user what to do. The options are:\\
\texttt{(r)ight, (l)eft, (t)urn, (q)uit}\\
and the user only has to type in the first letter of the available commands, followed by pressing Enter.

You then implement these actions using the functions that you have already built.
And you're done! Now you can sit back and quietly reflect on this moment of synergy, as you test your delightful little prototype.

\section*{Input}
The input starts with one line containing a string comprised of one character ``\texttt{r}'', ``\texttt{l}'', ``\texttt{t}'' or ``\texttt{q}''.
This input repeats until the input is ``\texttt{q}'' at which point the program exits.

\section*{Output}
First the program should print out the current state of the tetris piece.
Then the program should prompt the user ``\texttt{(r)ight, (l)eft, (t)urn, (q)uit: }'', without quotations.
This repeats until the user inputs ``q''.

