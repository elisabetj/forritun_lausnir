\problemname{Printer from Hell}

\illustration{0.4}{hell}{printer from hell}

One day, your printer loses the plot and starts printing text from the bottom left corner and upwards,
rather than from the top left corner and to the right.
Yet the characters are correctly oriented. When you attempt to print out:\\
%\begin{verbatim}
\texttt{Hey printer,}\\
\texttt{what's wrong}\\
\texttt{with you?}\\
%\end{verbatim}

\noindent
it comes out as:\\
\begin{verbatim}
,g 
rn 
eo 
tr?
nwu
i o
rsy
p'
 th
yat
ehi
Hww
\end{verbatim}

This puts you in a bit of a pickle. You have a small programming assignment that you need to hand in.
The professor is old and eccentric and insists that his students print out their solutions.
You need to find a way to correct for your printer's dementia.
You realize that if you ``rotate'' your text $90$ degrees clockwise before printing it,
then it prints out correctly.

Write a function named \texttt{rotate\_text\_clockwise(text)} that takes one string as an argument, 
\texttt{text}, and returns that \texttt{text} rotated clockwise $90$ degrees clockwise.

\textbf{Note : please only submit your function definitions, without any code outside the functions!}

\section*{Input}
The input consists of $n$ lines where $1 \leq n \leq 100$.
Each line will contain a string of at most $100$ symbols.
It is guaranteed that the string will only be composed of
letters from the English alphabet, spaces, and the following special characters:\\
\texttt{, ' ? !}.\\
Furthermore, no line will contain trailing spaces.

\section*{Output}
The output should consist of the input rotated $90$ degrees clockwise.
Ensure that each line in the output contains no trailing spaces
as your output must be exactly correct, including each space and newline character.
