\problemname{Pulling Levers for Coins}

Add the following functionality to tiles $(1,2)$, $(2,2)$, $(2,3)$ and $(3,2)$:

When the player enters these tiles, offer them the choice to pull a lever.
If they choose to pull the lever,
they are to be told that they received one coin.

There is no limit on how often the player can get a coin from each lever.
But they should only be presented with the option to pull a lever
upon entering the room.
So if they want to be asked again, they must first go away, to another tile, and come back.
Picking an invalid direction (walking into a wall) does not count as leaving the room.

In addition, you need to keep track of the number of coins the player has
and display that, each time a lever is pulled.
When the goal tile has been reached,
the program also prints out how many coins the player retrieved.

\section*{Input}

The input consists of a sequence $l = (l_1, \dots, l_n)$ of $n$ lines,
where each line $l_i$ should be interpreted as containing
either an answer to the question where the player wants to move
or an answer to the question whether the player wishes to pull a lever,
depending on the state of the game at the time when that line is encountered.

In the tests, the last line of input will always result in
the player reaching the goal tile.
It is guaranteed that the player will reach the goal tile
in at most $200$ moves.

\section*{Output}

The quick way to see how the output should be,
is to have a look at the given samples.
But if you run into any problems,
the following elaboration should help clarify.

\subsection*{More detailed output description}

The output consists of $m$ lines,
which form a pattern consisting of the following building blocks:
\begin{itemize}
    \item
    A direction block $D$, consisting of $2$ to $3$ lines:
    \begin{itemize}
        \item
        A direction menu $M_D$.
        \item
        A direction prompt $P_D$.
        \item
        An invalid direction alert $A_I$ (conditional).
    \end{itemize}
    \item
    A lever block $L$, consisting of $1$ to $2$ lines:
    \begin{itemize}
        \item
        A lever prompt $P_L$.
        \item
        An income report $R_I$ (conditional).
    \end{itemize}
    \item
    A victory report $R_V$.
\end{itemize}
A more detailed description of these blocks is given below,
but first we will describe the overall pattern.

\begin{itemize}
    \item
    The first block should be a direction block.
    \item
    The last block (and only the last block) should be a victory report.
    \item
    In between should be a sequence of direction blocks
    sometimes interleaved with a lever block.
    \item
    There should never be two lever blocks in a row,
    without a direction block between them.
    \item
    Any direction block containing an invalid direction alert
    should be immediately followed by another direction block.
    \item
    A lever block should never be preceded by
    a direction block containing an invalid direction alert.
    \item
    However, the absence of an invalid direction alert
    does not automatically mean that a lever block should follow.
    \item
    A lever block should only be presented
    when the corresponding game state,
    as kept track of by the program,
    indicates that the player is located
    on a tile with a lever.
\end{itemize}
So, the output should form a sequence $b = (b_1, \dots, b_k)$ of $k$ blocks,
where for each $i$ with $1 \le i < k$, the following rules should be observed:
\begin{itemize}
    \item
    If $b_i$ is a lever block or a direction block containing an invalid direction alert,
    then $b_{i+1}$ should be a direction block.
    \item
    If $b_{i+1}$ is a lever block or victory report,
    then $b_i$ should be a direction block without an invalid direction alert.
\end{itemize}
Here is a more detailed description of these blocks:
\begin{itemize}
    \item
    A direction block $D$ always begins with
    a direction menu, followed by a direction prompt.
    \begin{itemize}
        \item
        A direction menu $M_D$ consists of the line
        ``\texttt{You can travel: \{$v$\}.}'', without quotations,
        where $v$ is a list of the directions available
        from the current tile,
        separated by spaces and the word ``\texttt{or}''.
        In the presentation, the valid directions should be capitalized
        with the first letter inside parentheses:
        \texttt{(N)orth}, \texttt{(E)ast},
        \texttt{(S)outh}, and \texttt{(W)est}.

        For example, if the available directions are
        north, south and west,
        the direction menu consists of the line
        ``\texttt{You can travel: (N)orth or (S)outh or (W)est.}'',

        \item
        A direction prompt $P_D$ just consists of the line
        ``\texttt{Direction: }'', without quotations.
        The space at the end is optional,
        but there needs to be at least some whitespace
        to separate this line from the next line of the output.
        Of course, as long as the next ``line'' of the output does not appear on the same line,
        it is already separated by a carriage return (new line character).
    \end{itemize}
    Then, depending on the corresponding answer from the input,
    line $l_i$ for block $b_i$,
    an invalid direction alert should follow
    if the input $l_i$ does not correspond to a valid direction selection.
    The comparison should not be case-sensitive.
    The total set of valid choices is
    \texttt{N}, \texttt{E}, \texttt{S}, and \texttt{W},
    as well as their lower-case counterparts,
    but depending on the location of the player,
    the options will be restricted to various subsets of these choices.
    \begin{itemize}
        \item
        An invalid direction alert $A_I$ consists of the line
        ``\texttt{Not a valid direction!}''
    \end{itemize}

    \item
    A lever block $L$ always begins with a lever prompt.
    \begin{itemize}
        \item
        A lever prompt $P_L$ just consists of the line
        ``\texttt{Pull a lever (y/n): }'', without quotations.
        The space at the end is optional,
        as with the direction prompt.
    \end{itemize}
    Then, depending on the corresponding answer from the input,
    line $l_i$ for block $b_i$,
    an income report should follow if the input $l_i$ is ``\texttt{y}'',
    where the comparison should not be case-sensitive.
    \begin{itemize}
        \item
        An income report $R_I$ consists of the line
        ``\texttt{You received \{$c$\}, your total is now \{t\}.}'',
        where $c$ is the amount of coins received from the lever,
        and $t$ is the total amount of coins the player has received so far,
        since starting out at the starting location.
    \end{itemize}

    \item
    A victory report $R_V$ consists of the line
    ``\texttt{Victory! Total coins \{$T$\}.}'', without quotations,
    where $T$ is the total amount of coins the player received
    throughout their journey from the starting location to the goal tile.
\end{itemize}

As should be clear from the above,
$k$ should be equal to $n+1$,
that is, the number of blocks that form the output
should be one more than the lines of input,
as each block will correspond to (and ``consume'') one line of input,
except the final block, the victory report.
