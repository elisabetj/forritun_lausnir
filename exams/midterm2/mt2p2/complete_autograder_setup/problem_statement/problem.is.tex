\problemname{Tokenization (40\%)}

Skrifið forrit sem biður notandann að slá inn nafn á skrá sem inniheldur runu af orðum í sérhverri línu.
Forritið tilreiðir textann, þ.e. brýtur runurnar upp í tóka með því að nota hvít bil sem afmarkara (e. delimiter).
Ef komma, punktur, upphrópunarmerki eða spurningamerki (, . ! ?) kemur fyrir í lok orðs þá eru þessi greinarmerki meðhöndluð sem sérstakir tókar.
Gera má ráð fyrir að þetta séu einu mögulegu greinarmerkin og ef þau koma fyrir í inntaksskrá þá er það í lok orðs (ekkert hvítt bil er á undan greinarmerki).

\textbf{Ekki} er leyfilegt að nota import setningu í lausninni nema \texttt{import typing}. \\

Fyrsta dæmi um inntaksskrá, \texttt{data1.txt}:
\begin{verbatim}
One Two Three
Four Five Six
Seven Eight Nine
\end{verbatim}

Annað dæmi um inntaksskrá, \texttt{data2.txt}:
\begin{verbatim}
One Two Three? Four
Five, Six Seven Eight!
Nine Ten.
\end{verbatim}

Þriðja dæmi um inntaksskrá, \texttt{data3.txt}:
\begin{verbatim}
This is a text, 
with some
punctuation characters! attached 
to some words. 
Right?
\end{verbatim}


\section*{Inntak}
Inntakið er nafn skrár sem á að tilreiða, t.d. \texttt{data1.txt} eða hvaða annað skráarnafn sem er sem endar á \texttt{.txt}.
Inntaksskrá inniheldur $1$ til $10$ línur, þar sem sérhver lína inniheldur runu af $1$ til $15$ orðum, og lengd sérhvers orðs er $1$ til $10$ stafir.
Hér er orð skilgreint sem runa af hástöfum og/eða lágstöfum, mögulega með kommu, punkti, upphrópunarmerki eða spurningarmerki í lok orðsins.

Athugið: Forritið ykkar á ekki að skoða sérstaklega hvort þessi inntaksskilyrði standist, né hafna öðru inntaki. 
Þetta er bara ykkur til upplýsinga um það inntak sem er prófað í prófunartilvikunum. 
Þið þurfið sem sagt ekki að búast við inntaki utan þessara takmarkana.

\section*{Úttak}
Ef ekki er hægt að opna inntaksskrána þá er ekkert úttak myndað.
Annars samanstendur úttakið af eftirfarandi:
\begin{enumerate}
    \item Fjöldi orða í sér línu og sérhvert orð úr inntaksskránni í sér línu þar á eftir.
    \item Fjöldi tóka í sér línu og sérhver tóki úr inntaksskránni í sér línu þar á eftir. 
\end{enumerate}
