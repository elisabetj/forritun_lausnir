\problemname{Height (20\%)}

In this project, you implement the class \texttt{Height}, in the file \texttt{height.py}, which encapsulates height in feet and inches. 
By inspecting the main program below and its output, you should be able to figure out which instance variables and which methods are necessary in \texttt{Height}. 

You are \textbf{not} allowed to use any import statement in your solution, except \texttt{import typing}. \\

\noindent
Info on feet and inches:
\begin{itemize}
\item The foot (pl. feet; abbreviation: ft) is a unit of length in the imperial and US customary systems of measurement.
\item Since the International Yard and Pound Agreement of 1959, one foot is defined as 30.48 centimeters exactly.
\item In customary and imperial units, the foot comprises 12 inches. 
\item There are 2.54 centimeters in an inch.
\end{itemize}


\section*{Example main program:}

\begin{verbatim}
from height import Height

h1 = Height(5, 9)    # 5 feet, 9 inches
print(h1)
h2 = Height(5, 11)   # 5 feet, 11 inhces
print(h2)

c1 = h1.cm()        # converts h1 to centimeters
print(c1,"cm")

c2 = h2.cm()        # converts h2 to centimeters
print(c2,"cm")

print(h2 > h1)      
print(h1 > h2)

h3 = h1 + h2        # adds to heights
print(h3)
c3 = h3.cm()
print(c3,"cm")

h4 = Height(5, 12)
print(h4)
c4 = h4.cm()
print(c4,"cm")

print(h4 > h3)
\end{verbatim}

This main program is provided for your convenience.
You can download it and run it against your solution for testing locally.


\pagebreak
\section*{Output}
\noindent
Output from the main program above:
\begin{verbatim}
5 ft, 9 in
5 ft, 11 in
175 cm
180 cm
True
False
11 ft, 8 in
356 cm
6 ft, 0 in
183 cm
False
\end{verbatim}
\textbf{Note:} The cm() method should return the number of centimeters, rounded to the nearest integer.

\section*{Unit tests}
Your implemenation of \texttt{Height} will be tested with unit tests.
When the tests generate an instance of \texttt{Height}, the following values will be used:
\begin{itemize}
\item $1 \le feet \le 7$
\item $1 \le inches \le 36$ 
\end{itemize}

\textbf{Note}: Your implemenation is not supposed to validate this input, or refuse other input.
This is just for your information about the input in the test cases. 
You do not need to expect input that does not meet these restrictions.

\section*{Submission}
You should only submit the file \texttt{height.py}.
