\problemname{Spjótkast (40\%)}

Í þessu verkefni útfærið þið forrit sem les nöfn spjótkastara og lengd kasta þeirra (í metrum) í tiltekinni keppni.
Þessar upplýsingar eru lesnar úr skrá og prentaðar út á ákveðinn hátt (sjá fyrir neðan).

\noindent
Þið megið \textbf{ekki} nota neina import setningu í lausninni ykkar fyrir utan \texttt{import typing}. \\

\section*{Inntak}
Inntakið er ein lína sem inniheldur nafn á inntaksskrá sem endar á \texttt{.txt}.
Dæmigerð inntaksskrá, \texttt{throws1.txt}, lítur svona út:
\begin{verbatim}
Sonia Bisset 68.79
Lillian Copeland 72.28
Babe Didrikson 69.34
Tiina Lillak 65.46
\end{verbatim}

Annað dæmi um inntaksskrá, \texttt{throws2.txt}, er svona:
\begin{verbatim}
Leryn Franco 58.12
Ruth Fuchs 66.23
Trine Hattestad 61.35
Tiina Lillak 69.96
Kirsten Hellier 59.37
Haruka Kitaguchi 70.23
Tiina Lillak 71.69
Mirela Manjani 58.34
Kirsten Hellier 66.45
Tiina Lillak 63.44
Ruth Fuchs 71.68
Haruka Kitaguchi 62.27
\end{verbatim}

Þessar tvær skrár eru gefnar, ykkur til þæginda.
Þið getið hlaðið þeim niður til að prófa á eigin vél.

Inntaksskráin inniheldur alltaf þrjá dálka: fornafn keppanda, eftirnafn keppanda og lengd kasts viðkomandi keppanda.
Gera má ráð fyrir eftirfarandi atriðum varðandi inntaksskrána:
\begin{itemize} 
\item Hún inniheldur a.m.k. eina línu.
\item Fyrir sérhvern keppanda er fjöldi lína í mesta lagi fjórir.
\item Lengd nafns sérhvers keppanda er minni en $20$ stafir.
\item Kast $t$ er alltaf gild rauntala, þar sem $50 \le t \le 75$, með í mesta lagi tvo aukastafi.
\item Fjöldi umferða (lína í skránni) eru ekki endilega sá sami fyrir alla keppendur.  
\end{itemize}

\textbf{Athugið}: Forritið ykkar á ekki að skoða sérstaklega hvort þessi inntaksskilyrði standist, né hafna öðru inntaki. 
Þetta er bara ykkur til upplýsinga um það inntak sem er prófað í prófunartilvikunum. 
Þið þurfið sem sagt ekki að búast við inntaki utan þessara takmarkana.


\pagebreak
\section*{Úttak}
Ef ekki er hægt að opna inntaksskrána þá er úttakið tómt.
Annars samanstendur úttakið af einni línu fyrir sérhvern keppanda.
Sérhver lína samanstendur af eftirfarandi (í þessari röð):
\begin{itemize}
\item Nafni keppanda, vinstri jafnað í sviði með breiddina $20$.
\item Strengnum \texttt{"Throws: "}.
\item Lengdum einstakra kasta viðkomandi keppanda, með einu bili á milli þeirra.
\end{itemize}
Upplýsingarnar eru prentaðar út í sömu röð og gögnin koma fyrir í inntaksskránni.

Ef keppandi á sér fleiri en eitt kast, þá bætist ein lína við úttakið:
Hún inniheldur nafn og meðallengd kasta
þess keppanda sem á fleiri en eitt kast og á jafnframt hæsta meðaltalið.
Með öðrum orðum, látum $K$ vera þann keppanda sem á hæsta meðaltalið,
af öllum þeim keppendum sem eiga fleiri en eitt kast
(hunsið þá keppendur sem eiga bara eitt kast - eitt happaskot er ekki áreiðanlegur mælikvarði á hæfni þeirra).
Þá á þessi síðasta lína að innihalda nafn keppandans $K$,
tvípunkt, eitt bil, og síðan meðalkastlengd keppandans.
Meðaltalið skal námunda með tveimur aukastöfum.
Ef tveir eða fleiri keppendur eru jafnir um hæsta meðaltalið
þá skal birta þann sem kemur fyrst fyrir í inntaksskránni.

