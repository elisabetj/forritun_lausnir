\problemname{Javelin Throws (40\%)}

In this project, you implement a program which reads the names of Javelin competitors and their distances (in meters) thrown in some competition.
The information is read from a file and printed out in a specific manner (see below).

\noindent
You are \textbf{not} allowed to use any import statement in your solution, except \texttt{import typing}. \\

\section*{Input}
The input is one line containing the name of an input file, ending in \texttt{.txt}.
A typical input file, \texttt{throws1.txt}, looks like this:
\begin{verbatim}
Sonia Bisset 68.79
Lillian Copeland 72.28
Babe Didrikson 69.34
Tiina Lillak 65.46
\end{verbatim}

Another example of an input file, \texttt{throws2.txt}, looks like this:
\begin{verbatim}
Leryn Franco 58.12
Ruth Fuchs 66.23
Trine Hattestad 61.35
Tiina Lillak 69.96
Kirsten Hellier 59.37
Haruka Kitaguchi 70.23
Tiina Lillak 71.69
Mirela Manjani 58.34
Kirsten Hellier 66.45
Tiina Lillak 63.44
Ruth Fuchs 71.68
Haruka Kitaguchi 62.27
\end{verbatim}

Those two files are provided for your convenience.
You can download them for testing locally.

The input file always contains three columns: the first name of a competitor, the last name of the competitor, and the distance thrown by that competitor in a particular round of a competition.  
The following can be assumed about the input file:
\begin{itemize} 
\item It contains at least one line.
\item The number of lines for any given competitor is four at the maximum.
\item The length of the name of a competitor is less than $20$ characters.
\item The throw $t$ is always a valid real number, where $50 \le t \le 75$, with at most two decimal digits after the period.
\item The number of rounds (lines in the file) are not necessarily the same for all competitors.  
\end{itemize}

\textbf{Note}: Your program is not supposed to validate this input, or refuse other input.
This is just for your information about the input in the test cases. 
You do not need to expect input that does not meet these restrictions.


\pagebreak
\section*{Output}
If the input file cannot be opened, no output is generated.
Otherwise, the output consists of one line for each competitor.
Each line is composed of the following (in this order):
\begin{itemize}
\item The name of a competitor, left-justified in a field of width $20$.
\item The string \texttt{"Throws: "}.
\item The lengths of the individual throws for the competitor, separated by one space.
\end{itemize}
The information is printed in the same order as the data appears in the input file.

If any competitor has more than one throw, the output contains one more line:
It contains the name and the average throw length
of the competitor who has more than one throw and the highest average.
In other words, let $C$ be the competitor who has the highest average,
out of all those that have more than one throw
(ignore those that have only one throw - one lucky throw is not a reliable measure of their talent).
This last line should contain the name of the competitor $C$,
followed by a colon, one space, and then that competitor's average throw length.
The average throw should be rounded to two decimal digits.
If two or more players are tied for the highest average,
the name that appears first in the input file is displayed.

