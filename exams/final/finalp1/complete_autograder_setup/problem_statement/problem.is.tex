\problemname{Yatzy (40\%)}

Í þessu verkefni útfærið þið forrit til að spila mjög einfalda útgáfu af Yatzy.
Þessi útgáfa inniheldur eftirfarandi virkni:
\begin{itemize}
    \item Það er aðeins einn spilari.
    \item Það eru fimm teningar og sérhver þeirra er með gildi á bilinu 1 til 6.
    \item Hermt er eftir teningakasti með því að slá inn línu sem inniheldur fimm heiltölur.
    Forritið hættir keyrslu þegar tóm lína er slegin inn.
    \item Spilarinn fær punkta fyrir ákveðnar samsetningar á teningunum.
    Punktarnir $p$ eru reiknaðir á eftirfarandi hátt:
    \begin{enumerate}
        \item \textbf{Yatzy}: Allir $5$ teningarnir hafa sama gildið. Þá er $p = 50$.
        \item \textbf{Þrír eins}: Sama gildið kemur upp á $3$ teningum.
        Lát $v$ vera gildið sem kemur upp á þremur teningum. Þá er $p = 3 \cdot v$.
        Athugið að það er ekki nauðsynlegt að hinir tveir teningarnir séu með annað gildi en þessir þrír,
        en \textbf{Þrír eins} á samt aðeins við þegar ekki er um að ræða Yatzy.
        \item \textbf{Par}: Sama gildið kemur fyrir á $2$ teningum.
        Lát $v$ vera hæsta gildið sem kemur fyrir á teningapari. Þá er $p = 2 \cdot v$.
        M.ö.o., ef tvö pör koma fyrir á teningunum þá skal velja parið sem gefur hærri punkta.
        Par á aðeins við ef \textbf{Þrír eins} eru ekki til staðar.
        Ávallt skal velja \textbf{Þrír eins} fram yfir \textbf{Par} þrátt fyrir að það síðarnefnda gefi hærri punkta.
        \item Ef ekkert af ofangreindu á við, þ.e. ef allir teningarnir hafa mismundandi gildi, þá er $p=0$.
    \end{enumerate}
    \item Sérhvert teningakast fær stig óháð því hvaða samsetning hefur komið áður upp (sjá Sample 2 fyrir neðan).
\end{itemize}

Skoðið hinn gefna \textit{starter code}.
Hann inniheldur fallið \texttt{get\_counts()}
sem telur hversu oft sérhvert gildi kemur fyrir í teningakasti.
Þið \textbf{megið} nota þetta fall ef þið viljið.

\textbf{Ekki} er leyfilegt að nota import í lausninni nema \texttt{import typing}.


\section*{Inntak}
% First, a general description of the input,
% explaining to students how to interpret
% what they are seeing in the samples below.

Eins og sjá má af sýnidæmunum hér fyrir neðan,
þá inniheldur sérhver lína $5$ tölur, aðskildar með bilum.
Þær tákna útkomur þegar $5$ teningum er kastað.
Inntakið endar síðan á einni auðri línu
sem gefur til kynna að inntakið sé búið.
Sú lína sést ekki í reitunum með sýnidæmunum,
en hún er samt sem áður til staðar í inntakinu.

\subsection*{Formleg skilgreining á inntakinu}
% Then a more precise description, leaving nothing to doubt.

Nánar tiltekið samanstendur inntakið af $n + 1$ línum.
Síðasta línan er auð, en fyrstu $n$ línurnar innihalda niðurstöður úr teningaköstum.
Köllum þær $l_1, l_2, \dots, l_n$.

Sem sagt, fyrir sérhvert $j\in \{1, 2, \dots, n\}$,
inniheldur línan $l_j$ runu $v_j=(v_{j1}, v_{j2}, v_{j3}, v_{j4}, v_{j5})$
af $5$ heiltölum sem standa fyrir gildi teninga.
Eitt bil er á milli þessara gilda.

\subsection*{Takmarkanir prófunargilda}
% And finally, assurances about what to expect from the test cases.
% Here we mention what we will restrict the input to in the tests,
% to clarify that they do not need to worry about input outside those restrictions.
% This does not mean their solutions should validate the input,
% or reject input that does not satisfy those restrictions.

Í prófunartilvikunum verður $n$ takmarkað við $0\le n \le 20$,
og teningagildin verða heiltölur takmarkaðar við $1 \le v_{ji} \le 6$
fyrir öll $j\in \{1, \dots, n\}$ og $i\in \{1, \dots, 5\}$.

\textbf{Athugið}: Forritið ykkar á ekki að skoða sérstaklega hvort þessi inntaksskilyrði standist, né hafna öðru inntaki.
Þetta er bara ykkur til upplýsinga um það inntak sem er prófað í prófunartilvikunum. 
Það er gott ef forritið ykkar getur ráðið við önnur inntök líka,
en þið þurfið sem sagt ekki að búast við inntaki utan þessara takmarkana.


\pagebreak
\section*{Úttak}
% First, a general description of the output,
% explaining to students how to interpret
% what they are seeing in the samples below.
Fyrir sérhvert teningakast af $5$ teningunum er ein lína í úttakinu
sem inniheldur punktana sem fengust fyrir kastið eins og tilgreindir eru að ofan.

\subsection*{Formleg úttakslýsing}
% Then a more precise description, leaving nothing to doubt.

Nánar tiltekið á úttakið að samanstanda af $n$ línum,
köllum þær $\lambda_1, \dots, \lambda_n$.

Fyrir hvert $j\in \{1, \dots, n\}$
á línan $\lambda_j$ að innihalda eina heiltölu $p_j$,
sem á að vera fjöldi punkta sem leikmaðurinn fékk
fyrir teningakastið í línu $l_j$ í inntakinu.
