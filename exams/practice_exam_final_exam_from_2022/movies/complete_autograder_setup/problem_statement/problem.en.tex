\problemname{Movies}

Write a program, \texttt{movies.py}, that extracts data from a file.
The file contains information about top $N$ movies according to ratings given by viewers.
The program should allow the user to perform certain operations related to the data.
An input file, for the top $20$ movies \texttt{movies-top-20.csv}, looks like this:

\begin{verbatim}
The Shawshank Redemption;9.2;1994
The Godfather;9.1;1972
The Godfather: Part II;9.0;1974
Il buono, il brutto, il cattivo;8.9;1966
Pulp Fiction;8.9;1994
Inception;8.9;2010
Schindler's List;8.9;1993
12 Angry Men;8.9;1957
One Flew Over the Cuckoo's Nest;8.8;1975
The Dark Knight;8.8;2008
Star Wars: Episode V - The Empire Strikes Back;8.8;1980
The Lord of the Rings: The Return of the King;8.8;2003
Shichinin no samurai;8.8;1954
Star Wars;8.7;1977
Goodfellas;8.7;1990
Casablanca;8.7;1942
Fight Club;8.7;1999
Cidade de Deus;8.7;2002
The Lord of the Rings: The Fellowship of the Ring;8.7;2001
Rear Window;8.7;1954
\end{verbatim}

A semicolon is used as as separator between the columns in the file.  The first column contains the movie title, the second column the rating, and the third column the year of the release.
The program displays a menu, allowing the user to perform the following operations:
\begin{itemize}
    \item Display information about the movies, alphabetically ordered by title.
    \item Display all movie titles released in a given year.
    \item Change the rating of all the movies by a certain number (note, however, that the input file is not changed).
\end{itemize}

Note that it should be easy to change the letters that stand for the menu options, e.g. using \texttt{a}, \texttt{b}, and \texttt{c} instead of \texttt{1}, \texttt{2}, and \texttt{3}.

One function, \texttt{open\_file()} is given which you \textbf{must} use, and \textbf{cannot} change.

\section*{Input}
The first line of input contains the name of the file.
If the file does not exist, no more input will follow and the program exits immediately.

Then menu item selection followg, which starts with a line containing one of the following options:
\begin{itemize}
    \item If option \texttt{1} is selected, then the menu selection will repeat with no input between.
    \item If option \texttt{2} is selected, then a line will follow containing one integer $y$, where $0 \leq y \leq 3000$, the release year for which movies should be listed.
    \item If option \texttt{3} is selected, then a line will follow containing one floating point number $x$, where $-10 \leq x \leq 10$, the change in rating to apply to all movies.
    \item If any other menu option than \texttt{1}, \texttt{2}, or \texttt{3} is entered, no more input will follow and the program quits running.
\end{itemize}

The menu item selection repeats until the program exits.

\section*{Output}
The program should prompt 
The menu appears as follows:
\begin{verbatim}
*******************************
1. Movies in alphabetical order
2. Titles in given year
3. Modify all ratings
*******************************
\end{verbatim}

See the samples for information on output within each menu item.

The number of \texttt{*} symbols written out in the menu is $31$.
The field for the movie title is left justified with width $50$.
The field for the movie rating is right justified with width $6$ and exactly $2$ digits after the decimal point.
The field for the year is right justified with width $6$.
