\problemname{Team}
(15 stig; 25\%)

In this project, you need to implement two classes:
\begin{itemize}
    \item \texttt{Player}, in the file \texttt{player.py}, which encapsulates a (soccer) player. 
    One of the public instance variables in Player is goals which stores the number of goals that the player has scored.
    \item \texttt{Team}, in the file \texttt{team.py}, which encapsulates a (soccer) team.
    By inspecting the given main program and the corresponding output, you should be able to figure out which variables and methods the two classes need.

The output from the main program:
\begin{verbatim}
Mohamed Salah, Goals: 5
Roberto Firmino, Goals: 1
Luis Díaz, Goals: 4
Marcus Rashford, Goals: 4
Harry Maguire, Goals: 5
Christiano Ronaldo, Goals: 1
Liverpool:
        Mohamed Salah, Goals: 5
        Luis Díaz, Goals: 4
        Roberto Firmino, Goals: 1
Manchester United:
        Harry Maguire, Goals: 5
        Marcus Rashford, Goals: 4
        Christiano Ronaldo, Goals: 1
Mohamed Salah, Goals: 5
Harry Maguire, Goals: 5
Liverpool+Manchester United:
        Mohamed Salah, Goals: 5
        Harry Maguire, Goals: 5
        Luis Díaz, Goals: 4
        Marcus Rashford, Goals: 4
        Roberto Firmino, Goals: 1
        Christiano Ronaldo, Goals: 1
Mohamed Salah, Goals: 5
\end{verbatim}


\textbf{Note} that when a team is printed out its players are shown in descending order on the number of goals they have scored.
If two or more players have scored equal number of goals, the one that was added first to the team is shown first. 
One tab-character is printed in each line before the information for each player is shown.

Let us assume that players is some kind of a collection of instances of the Player class. In order to sort players in descending order on goals scored, you can use:
\texttt{sorted(players, key=lambda p: p.goals, reverse=True)}

Here, ``\texttt{lambda p: p.goals}'' is an anonymous function which returns the value of the goals variable in the instance $p$.
