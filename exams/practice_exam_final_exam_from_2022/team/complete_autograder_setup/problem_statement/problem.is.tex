\problemname{Team}
(15 stig; 25\%)

Í þessu verkefni eigið þið að útfæra tvo klasa:
\begin{itemize}
    \item \texttt{Player}, í skránni \texttt{player.py}, sem hjúpar (fótbolta)leikmann. Ein af public tilvikabreytum í \texttt{Player} er goals sem geymir fjölda marka sem viðkomandi leikmaður hefur skorað.
    \item \texttt{Team}, í skránni \texttt{team.py}, sem hjúpar (fótbolta)lið.
\end{itemize}
Með því að skoða gefna aðalforritið hér fyrir ásamt úttaki þess hér að neðan eigið þið að geta áttað ykkur á því hvaða breytur og aðgerðir klasarnir tveir þurfa að hafa.


Úttak aðalforrits:
\begin{verbatim}
Mohamed Salah, Goals: 5
Roberto Firmino, Goals: 1
Luis Díaz, Goals: 4
Marcus Rashford, Goals: 4
Harry Maguire, Goals: 5
Christiano Ronaldo, Goals: 1
Liverpool:
        Mohamed Salah, Goals: 5
        Luis Díaz, Goals: 4
        Roberto Firmino, Goals: 1
Manchester United:
        Harry Maguire, Goals: 5
        Marcus Rashford, Goals: 4
        Christiano Ronaldo, Goals: 1
Mohamed Salah, Goals: 5
Harry Maguire, Goals: 5
Liverpool+Manchester United:
        Mohamed Salah, Goals: 5
        Harry Maguire, Goals: 5
        Luis Díaz, Goals: 4
        Marcus Rashford, Goals: 4
        Roberto Firmino, Goals: 1
        Christiano Ronaldo, Goals: 1
Mohamed Salah, Goals: 5
\end{verbatim}


\textbf{Athugið} að þegar lið er prentað út þá birtast leikmenn þess í lækkandi röð miðað við fjölda marka sem þeir hafa skorað.
Ef tveir eða fleiri leikmenn hafa skorað jafnmörg mörk þá birtist sá fyrstur sem fyrst var bætt við liðið. 
Einn tab-character kemur í hverri línu á undan upplýsingum um sérhvern leikmann liðsins

Gefum okkur að players sé einhvers konar safn af tilvikum af Player klasanum . Til að raða players í lækkandi röð á skoruð mörk, þá getið þið notað:

\texttt{sorted(players, key=lambda p: p.goals, reverse=True)}

Hér er ``\texttt{lambda p: p.goals}'' nafnlaust fall sem skilar gildinu á goals breytunni í tilvikinu p.
