\problemname{Darts}


% \textbf{The main file, which handles input and output, is already provided. -
% Please only submit your function definitions, without any code outside the functions!}

Write a program, \texttt{darts.py},
which allows the user to repeatedly input the scores of two players in darts
until one of them wins.
\begin{itemize}
    \item
    The game is the so-called ``301 darts'',
    which means that both players start with a score of 301 points.

    \item
    Each player then takes alternating turns
    at throwing their darts at the dartboard
    and the points scored are subtracted from the total.

    \item
    The first player to reach zero points wins the game.

    \item
    If a player scores more points in a given round
    than the total points remaining,
    then the throw is invalid and the remaining total points are not changed.

    \item
    The entered scores need to be error checked,
    ensuring that they are integers.

    \item
    Constants are given which you MUST use.
\end{itemize}

Example input/output:

\begin{verbatim}
Input points for Player 1: 180
Player 1 remaining points: 121
Input points for Player 2: 60
Player 2 remaining points: 241
Input points for Player 1: 40
Player 1 remaining points: 81
Input points for Player 2: x
Invalid input!
Input points for Player 2: y
Invalid input!
Input points for Player 2: 100
Player 2 remaining points: 141
Input points for Player 1: 21
Player 1 remaining points: 60
Input points for Player 2: 40
Player 2 remaining points: 101
Input points for Player 1: 120
Player 1 remaining points: 60
Input points for Player 2: 40
Player 2 remaining points: 61
Input points for Player 1:
Invalid input!
Input points for Player 1: 60
Player 1 remaining points: 0
Player 1 won!
\end{verbatim}

Lines starting with ``Input'' are inputs, while all other lines are outputs.
Here it can be seen that the total remaining points for a player
are printed out in each round.
Moreover, an error message is printed when the input is invalid.
Finally, information about the winner is printed.

% \section*{Input}

% % First, a general description of the input,
% % explaining to students how to interpret
% % what they are seeing in the samples below.

% % \subsection*{Formal input specifications}
% % Then a more precise description, leaving nothing to doubt.

% % \subsection*{Test case restrictions}
% % And finally, assurances about what to expect from the test cases.
% % Here we mention what we will restrict the input to in the tests,
% % to clarify that they do not need to worry about input outside those restrictions.
% % This does not mean their solutions should validate the input,
% % or reject input that does not satisfy those restrictions.


% \section*{Output}

% % First, a general description of the output,
% % explaining to students how to interpret
% % what they are seeing in the samples below.

% % \subsection*{Formal output specifications}
% % Then a more precise description, leaving nothing to doubt.
