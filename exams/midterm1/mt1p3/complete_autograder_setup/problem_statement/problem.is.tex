\problemname{Prices}

\noindent
Skrifið forrit sem les inn runu af vöruverðum þangað til gildið $0$ er slegið inn.
Að lokum skrifar forritið út:
\begin{itemize} 
    \item Töluna $m$, fjölda vöruverða, sem slegin voru inn. Talan $0$ sem endar rununa er ekki talin með.
    \item Töluna $t$, heildarverð varanna.
    \item Töluna $l$, lægsta verðið, en þó eingöngu ef verð fyrir a.m.k. eina vöru var slegið inn.
\end{itemize}

Athugið: Munið að gefa breytunum ykkar lýsandi nöfn.
Stærðfræðilegu táknin sem eru notuð hér eru ekki tillögur að breytuheitum, heldur eru bara notuð til skýringa.

\section*{Inntak}
Inntakið samanstendur af $n$ línum, með $n \ge 1$. \\
Sérhver lína $i$, með $1 \le i \le n$, innheldur eina rauntölu $p_i$. \\
Síðasta talan er $p_n = 0$, en $p_i > 0$ fyrir öll $i < n$. \\
Óhætt er að gera ráð fyrir að $1 \le n \le 20$,
einnig að $0 \le p_i \le 1\,000$ fyrir öll $i$,
og ennfremur að sérhver rauntala $p_i$ í inntakinu sé gefin með í mesta lagi einum aukastaf.

Athugið: Forritið ykkar á ekki að skoða sérstaklega hvort þessi inntaksskilyrði standist, né hafna öðru inntaki.
Þetta er bara ykkur til upplýsinga, um það inntak sem er prófað í prófunartilvikunum.
Þið þurfið sem sagt ekki að búast við inntaki utan þessara takmarkana.

\section*{Úttak}
Úttakið samanstendur af í mesta lagi þremur línum á eftirfarandi sniði: \\ \\
\noindent
Number of items: <$m$> \\
\noindent
Total price: <$t$> \\
\noindent
Lowest price: <$l$> \\ \\
\noindent
Þriðja línan er eingöngu prentuð út ef a.m.k. eitt vöruverð var slegið inn. \\
\noindent
Gildið $t$ á að vera námundað með einum aukstaf með því að nota innbyggða \texttt{round()} fallið. \\
\noindent
Gildið $l$ skal prenta út með eins mörgum aukastöfum og þarf.
