\problemname{Prices}

\noindent
Write a program which reads in a sequence of prices of goods until the value $0$ is entered.
At the end, the program prints:
\begin{itemize} 
    \item The number $m$ of prices entered, not including the $0$ that ends the sequence.
    \item The total price $t$ of the goods.
    \item The lowest price $l$, but only if at least one price was input.
\end{itemize}

Note: Remember to use descriptive names for your variables.
The mathematical symbols used here are not suggestions for variable names, but just used for clarity.

\section*{Input}
The input consists of $n$ lines, with $n \ge 1$. \\
Each line $i$, for $1 \le i \le n$, contains one real number $p_i$. \\
The last one will be $p_n = 0$, but $p_i > 0$ for all $i < n$. \\
It is safe to assume that $1 \le n \le 20$,
and that $0 \le p_i \le 1\,000$ for all $i$,
and also that each number $p_i$ in the input is given with at most one digit after the decimal point.

Note: Your program is not supposed to validate this input, or refuse other input.
This is just for your information about the input in the test cases.
You do not need to expect input that does not meet these restrictions.

\section*{Output}
The output consists of maximum three lines formatted as follows: \\ \\
\noindent
Number of items: <$m$> \\
\noindent
Total price: <$t$> \\
\noindent
Lowest price: <$l$> \\ \\
\noindent
The third line is only printed if at least one price was input. \\
\noindent
The value $t$ should be rounded to one digit after the decimal point by using the built-in \texttt{round()} function. \\
\noindent
The value $l$ should be printed with the same precision as entered by the user.
