\problemname{Series}

\noindent
A geometric series is a sum of terms where the ratio of every two consecutive terms is the same.
The following is an example of such a series with $n=4$ terms and the ratio $r=\frac{1}{2}$:

\[
\frac{1}{2} + \frac{1}{4} + \frac{1}{8} + \frac{1}{16}
\]

\noindent
Write a program which uses a \textbf{for-loop} to compute the example series, given the length $n$ as input. \\
\noindent

Note: Remember to use descriptive names for your variables.
The mathematical symbols used here are not suggestions for variable names.

\section*{Input}
The input consists of one line with one integer $n$, where $0 \leq n \leq 10$.

Note: Your program is not supposed to validate this input, or refuse other input.
This is just for your information about the input in the test cases.
You do not need to expect input that does not meet these restrictions.

\section*{Output}
The output consists of $n$ lines, where the $i^{th}$ line shows the cumulative sum,
$\frac{1}{2} + \frac{1}{4} + \dots + \frac{1}{2^i}$, at each step $i$ in the iteration, with $1 \leq i \leq n$.
The $n^{th}$ line thus contains the sum of all the $n$ terms.
