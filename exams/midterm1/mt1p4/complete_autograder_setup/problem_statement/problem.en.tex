\problemname{Making Change}

\noindent
A clerk works in a store where the price of each item, given in dollars, is a positive integer.
For example, something might cost $\$21$, but nothing costs $\$9.99$. 
In order to make change, a clerk has an unbounded number of bills in each of the following denominations:
$\$1$, $\$2$, $\$5$, $\$10$, and $\$20$.

Write a program that reads the price of an item and the amount paid,
and prints how to make change using the smallest possible number of bills.

Note: Remember to use descriptive names for your variables.
The mathematical symbols used here are not suggestions for variable names.

\section*{Input}
The input consists of two lines:

The first line is an integer, $p$, denoting the price of an item, with $1 \leq p \leq 100$.

The second line is an integer, $a$, denoting the amount paid, with $p \leq a \leq 100$.

Note: Your program is not supposed to validate this input, or refuse other input.
This is just for your information about the input in the test cases.
You do not need to expect input that does not meet these restrictions.

\section*{Output}
The output shows how to make change using the smallest possible number of bills.

If there is no change to give, then no output is given.

Otherwise, each line in the output consists of an integer
$b \in \{1, 2, 5, 10, 20\}$,
denoting a bill of change to be given.
The bills should be given in descending order (largest first).
