\problemname{Making Change}

\noindent
Afgreiðslumaður vinnur í búð þar sem verð sérhverrar vöru í dollurum er jákvæð heiltala.
T.d. gæti einhver tiltekin vara kostað $\$21$, en engin vara hins vegar kostað $\$9.99$. 
Til að gefa til baka hefur afgreiðslumaðurinn ótakmarkað magn af seðlum í eftirfarandi upphæðum: 
$\$1$, $\$2$, $\$5$, $\$10$, og $\$20$.

Skrifið forrit sem les inn verð einnar vöru og greidda upphæð í dollurum,
og prentar út hvernig eigi að gefa til baka með því að nota sem fæsta seðla.

Athugið: Munið að gefa breytunum ykkar lýsandi nöfn.
Stærðfræðilegu táknin sem eru notuð hér eru ekki tillögur að breytuheitum.

\section*{Inntak}
Inntakið samanstendur af tveimur línum:

Fyrri línan er heiltala, $p$, sem stendur fyrir verð vöru, þar sem $1 \leq p \leq 100$.

Seinni línan er heiltala, $a$, sem stendur fyrir greidda upphæð, þar sem $p \leq a \leq 100$.

Athugið: Forritið ykkar á ekki að skoða sérstaklega hvort þessi inntaksskilyrði standist, né hafna öðru inntaki.
Þetta er bara ykkur til upplýsinga, um það inntak sem er prófað í prófunartilvikunum.
Þið þurfið sem sagt ekki að búast við inntaki utan þessara takmarkana.

\section*{Úttak}
Úttakið sýnir hvernig gefa eigi til baka með því að nota sem fæsta seðla.

Ef afgangurinn er enginn þá er ekkert úttak sýnt.

Annars inniheldur sérhver lína í úttakinu eina heiltölu
$b \in \{1, 2, 5, 10, 20\}$,
sem táknar seðil af skiptimynt.
Seðlarnir eiga að vera gefnir í lækkandi röð.
